%% Generated by Sphinx.
\def\sphinxdocclass{report}
\documentclass[letterpaper,10pt,ngerman]{sphinxmanual}
\ifdefined\pdfpxdimen
   \let\sphinxpxdimen\pdfpxdimen\else\newdimen\sphinxpxdimen
\fi \sphinxpxdimen=49336sp\relax

\usepackage[margin=1in,marginparwidth=0.5in]{geometry}
\usepackage[utf8]{inputenc}
\ifdefined\DeclareUnicodeCharacter
  \DeclareUnicodeCharacter{00A0}{\nobreakspace}
\fi
\usepackage{cmap}
\usepackage[T1]{fontenc}
\usepackage{amsmath,amssymb,amstext}
\usepackage{babel}
\usepackage{times}
\usepackage[Sonny]{fncychap}
\usepackage{longtable}
\usepackage{sphinx}

\usepackage{multirow}
\usepackage{eqparbox}

% Include hyperref last.
\usepackage{hyperref}
% Fix anchor placement for figures with captions.
\usepackage{hypcap}% it must be loaded after hyperref.
% Set up styles of URL: it should be placed after hyperref.
\urlstyle{same}

\addto\captionsngerman{\renewcommand{\figurename}{Abb.}}
\addto\captionsngerman{\renewcommand{\tablename}{Tab.}}
\addto\captionsngerman{\renewcommand{\literalblockname}{Quellcode}}

\addto\extrasngerman{\def\pageautorefname{Seite}}

\setcounter{tocdepth}{1}



\title{ManagementBinaryTrees Documentation}
\date{Feb. 09, 2017}
\release{0.1.0}
\author{Steffen Exler}
\newcommand{\sphinxlogo}{}
\renewcommand{\releasename}{Release}
\makeindex

\begin{document}
\if\catcode`\"\active\shorthandoff{"}\fi
\maketitle
\sphinxtableofcontents
\phantomsection\label{\detokenize{index::doc}}



\chapter{Einleitung}
\label{\detokenize{index:einleitung}}\label{\detokenize{index:managementbinarytrees-s-handbuch}}

\section{Über das Programm}
\label{\detokenize{intro:uber-das-programm}}\label{\detokenize{intro::doc}}
Das Programm ``ManagementBinaryTrees'' ist ein grafisches Programm, welches dazu dient Knoten in einem
Binärbaum ein zu fügen, zu verändern und zu entfernen.

Außerdem bietet es die Möglichkeit die Binärbaume
in eine Json Datei zu speichern und laden und zu sortieren.

Nach dem Start wird ein generierter Binärbaum angezeigt, der nach belieben bearbeite oder ausgetauscht
werden kann.

In den Menü welches sich unter den Button ``File'' verbirgt, ist es möglich Binärbaume zu laden, speicher,
neu an zu legen, zu sortieren und das Programm zu schließen.

Das Projekt wurde mit JUnit 4 tests getestete und die test Klassen befinden sich bei den Quellcode dabei.
\begin{itemize}
\item {} 
Quellcode: \url{https://github.com/linuxluigi/ManagementBinaryTrees}

\item {} 
Online Dokumentation: \url{https://github.com/linuxluigi/ManagementBinaryTrees/blob/master/docs/source/index.rst}

\end{itemize}

\index{Git}
\index{Quellcode}
\index{Online Dokumentation}
\noindent\sphinxincludegraphics{{ManagementBinaryTrees_01}.png}


\section{Kompilieren}
\label{\detokenize{intro:kompilieren}}
\index{Kompilieren}
\index{Build}
\index{Build Artifacts}
\index{Maven}
Das Projekt wurde via Maven 2 konstruiert und kann mit ein Konsolen Befehl in einer Jar Datei Kompiliert werden, dafür
muss aber zuerst Maven 2 installiert werden, unter Ubuntu / Debian muss folgendes eingeben werden.

\begin{sphinxVerbatim}[commandchars=\\\{\}]
\PYGZdl{} sudo apt\PYGZhy{}get install maven2
\end{sphinxVerbatim}

Jetzt wurde Maven 2 installiert und nun kann das Projekt die abhänigkeiten installiert werden, test ausgeführt und
zur einer ausführbaren Jar ausgeben.

\begin{sphinxVerbatim}[commandchars=\\\{\}]
\PYGZdl{} mvn clean install
\end{sphinxVerbatim}

\noindent\sphinxincludegraphics{{ManagementBinaryTrees_02}.png}


\section{Abhänigkeiten}
\label{\detokenize{intro:abhanigkeiten}}
\index{Abhänigkeiten}
Das Projekt wurde als Maven 2 Modul geschrieben und verwendet folgende Maven Module.

Maven Projekt Website: \url{https://maven.apache.org/}


\subsection{Google GSON}
\label{\detokenize{intro:google-gson}}
\index{Google GSON}
Gson ist eine Java Bibliothek die es ermöglicht Klassen und Variablen als Json Datei aus zu geben oder
ein String als Klasse oder Variable zu konvertieren.

\begin{DUlineblock}{0em}
\item[] Name: google-gson
\item[] Hersteller: Google Inc.
\item[] Version: 2.7
\item[] Link: \url{https://github.com/google/gson}
\end{DUlineblock}


\subsection{JUnit}
\label{\detokenize{intro:junit}}
\index{JUnit}
Junit ist ein unit testing Framework für Java von Erich Gamma und Kent Beck.

\begin{DUlineblock}{0em}
\item[] Name: JUnit
\item[] Hersteller: Erich Gamma und Kent Beck
\item[] Version: 4.12
\item[] Link: \url{http://junit.org/junit4/}
\end{DUlineblock}


\chapter{Bedienung}
\label{\detokenize{index:bedienung}}

\section{User Interface}
\label{\detokenize{interface::doc}}\label{\detokenize{interface:user-interface}}

\subsection{Bedienung}
\label{\detokenize{interface:bedienung}}
Nach dem Start des Programmes wird ein Beispiel Binärbaum erstellt, wobei jeder Knoten ein Button ist.
Diese Button dienen dazu die Knoten zu bearbeiten, neue ein zu fügen und zu löschen

\noindent\sphinxincludegraphics{{ManagementBinaryTrees_03}.png}

Nachdem ein Button gedrück worden ist erscheint ein extra Fenster welches die Optionen besitzt ``ADD'',
``RENAME'' und ``REMOVE''. Wohinter sich die Aktionen ``ADD'' \textgreater{}\textgreater{} Hinzufügen, ``RENAME'' \textgreater{}\textgreater{} ``Name Ändern'' und ``REMOVE'' \textgreater{}\textgreater{}
Knoten entfernen verbergen.

Das Textfeld gibt den Aktuellen inhalt des Knotens wieder. Wenn dieses geändert wird und auf ``RENAME'' geklickt wird,
ändert sich der Inhalt von den Knoten und wenn auf ``ADD'' geklickt wird, wird ein neuer Knoten mit den Wert der in dem
Textfeld enthalten ist hinzugefügt.

\noindent\sphinxincludegraphics{{ManagementBinaryTrees_04}.png}


\subsection{Knoten hinzufügen}
\label{\detokenize{interface:knoten-hinzufugen}}
Nachdem auf ein Knoten geklickt worden ist, ist es möglich in den Dialog Fenster, welches erscheint.
Den neuen Knoten ein 3 Stelligen Wert zu zu Ordnen und ihn nach dem angeklickten Knoten ein zu fügen.

\noindent\sphinxincludegraphics{{ManagementBinaryTrees_05}.png}

In diesen Beispiel erscheint nun unten Links der neue Knoten.

\noindent\sphinxincludegraphics{{ManagementBinaryTrees_06}.png}


\subsection{Binärbaum Sortieren}
\label{\detokenize{interface:binarbaum-sortieren}}
Um ein Binärbaum zu sortieren muss auf geklickt werden auf ``FILE'' \textgreater{}\textgreater{} ``Sort ACS'' oder ``Sort DECS''

\noindent\sphinxincludegraphics{{ManagementBinaryTrees_07}.png}

In diesen Beispiel sieht der Baum danach so aus:

\noindent\sphinxincludegraphics{{ManagementBinaryTrees_08}.png}


\subsection{Datei Laden und Sichern}
\label{\detokenize{interface:datei-laden-und-sichern}}
\index{Json Laden und Sichern}
Um ein Binärbaum zu sichern oder laden muss auf geklickt werden auf ``FILE'' \textgreater{}\textgreater{} ``Load File'' oder ``Save File''

\noindent\sphinxincludegraphics{{ManagementBinaryTrees_09}.png}

Und in den folgenden Fenster die gewünschte Json Datei auswählen.

\noindent\sphinxincludegraphics{{ManagementBinaryTrees_10}.png}

Hier ist eine Beispiel Json Datei.

\begin{sphinxVerbatim}[commandchars=\\\{\}]
\PYG{p}{[}
  \PYG{p}{[}\PYG{l+s+s2}{\PYGZdq{}\PYGZdq{}}\PYG{p}{,}\PYG{l+s+s2}{\PYGZdq{}L\PYGZdq{}}\PYG{p}{]}\PYG{p}{,}
  \PYG{p}{[}\PYG{l+s+s2}{\PYGZdq{}0\PYGZdq{}}\PYG{p}{,}\PYG{l+s+s2}{\PYGZdq{}K\PYGZdq{}}\PYG{p}{]}\PYG{p}{,}
  \PYG{p}{[}\PYG{l+s+s2}{\PYGZdq{}00\PYGZdq{}}\PYG{p}{,}\PYG{l+s+s2}{\PYGZdq{}K\PYGZdq{}}\PYG{p}{]}\PYG{p}{,}
  \PYG{p}{[}\PYG{l+s+s2}{\PYGZdq{}000\PYGZdq{}}\PYG{p}{,}\PYG{l+s+s2}{\PYGZdq{}G\PYGZdq{}}\PYG{p}{]}\PYG{p}{,}
  \PYG{p}{[}\PYG{l+s+s2}{\PYGZdq{}001\PYGZdq{}}\PYG{p}{,}\PYG{l+s+s2}{\PYGZdq{}G\PYGZdq{}}\PYG{p}{]}\PYG{p}{,}
  \PYG{p}{[}\PYG{l+s+s2}{\PYGZdq{}01\PYGZdq{}}\PYG{p}{,}\PYG{l+s+s2}{\PYGZdq{}J\PYGZdq{}}\PYG{p}{]}\PYG{p}{,}
  \PYG{p}{[}\PYG{l+s+s2}{\PYGZdq{}010\PYGZdq{}}\PYG{p}{,}\PYG{l+s+s2}{\PYGZdq{}F\PYGZdq{}}\PYG{p}{]}\PYG{p}{,}
  \PYG{p}{[}\PYG{l+s+s2}{\PYGZdq{}011\PYGZdq{}}\PYG{p}{,}\PYG{l+s+s2}{\PYGZdq{}D\PYGZdq{}}\PYG{p}{]}\PYG{p}{,}
  \PYG{p}{[}\PYG{l+s+s2}{\PYGZdq{}1\PYGZdq{}}\PYG{p}{,}\PYG{l+s+s2}{\PYGZdq{}K\PYGZdq{}}\PYG{p}{]}\PYG{p}{,}
  \PYG{p}{[}\PYG{l+s+s2}{\PYGZdq{}10\PYGZdq{}}\PYG{p}{,}\PYG{l+s+s2}{\PYGZdq{}I33\PYGZdq{}}\PYG{p}{]}\PYG{p}{,}
  \PYG{p}{[}\PYG{l+s+s2}{\PYGZdq{}100\PYGZdq{}}\PYG{p}{,}\PYG{l+s+s2}{\PYGZdq{}C\PYGZdq{}}\PYG{p}{]}\PYG{p}{,}
  \PYG{p}{[}\PYG{l+s+s2}{\PYGZdq{}101\PYGZdq{}}\PYG{p}{,}\PYG{l+s+s2}{\PYGZdq{}AA\PYGZdq{}}\PYG{p}{]}\PYG{p}{,}
  \PYG{p}{[}\PYG{l+s+s2}{\PYGZdq{}11\PYGZdq{}}\PYG{p}{,}\PYG{l+s+s2}{\PYGZdq{}G22\PYGZdq{}}\PYG{p}{]}\PYG{p}{,}
  \PYG{p}{[}\PYG{l+s+s2}{\PYGZdq{}110\PYGZdq{}}\PYG{p}{,}\PYG{l+s+s2}{\PYGZdq{}A\PYGZdq{}}\PYG{p}{]}
\PYG{p}{]}
\end{sphinxVerbatim}

Konvertiert in ein Binärbaum sieht es folgend aus:

\noindent\sphinxincludegraphics{{ManagementBinaryTrees_11}.png}


\chapter{Klassenbeschreibung}
\label{\detokenize{index:klassenbeschreibung}}

\section{Indexierung der Knoten}
\label{\detokenize{aboutIndex::doc}}\label{\detokenize{aboutIndex:indexierung-der-knoten}}
Das System für die Indexierung lautet:
\begin{enumerate}
\item {} 
Versuche nach links unten zu gehen.

\item {} 
Geht es nicht mehr nach links unten versuche nach rechts unten zu gehen.

\item {} 
Komme ich von Links unten, versuche ich nach rechts unten zu gehen.

\item {} 
Geht es nicht mehr nach rechts unten, gehe nach oben.

\item {} 
Wenn ich von ein Knoten rechts unten komme, gehe nach oben.

\item {} 
Wenn ich nicht mehr nach oben gehen kann, bin ich fertig.

\end{enumerate}

\noindent\sphinxincludegraphics{{BinaryTree}.png}


\section{ManagementBinaryTress}
\label{\detokenize{packages::doc}}\label{\detokenize{packages:managementbinarytress}}

\subsection{com.linuxluigi.edu}
\label{\detokenize{com/linuxluigi/edu/package-index:com-linuxluigi-edu}}\label{\detokenize{com/linuxluigi/edu/package-index::doc}}\label{\detokenize{com/linuxluigi/edu/package-index:package-com.linuxluigi.edu}}\index{com.linuxluigi.edu (package)}

\subsubsection{Controller}
\label{\detokenize{com/linuxluigi/edu/Controller::doc}}\label{\detokenize{com/linuxluigi/edu/Controller:controller}}\index{Controller (Java class)}

\begin{fulllineitems}
\phantomsection\label{\detokenize{com/linuxluigi/edu/Controller:com.linuxluigi.edu.Controller}}\pysigline{public class \sphinxbfcode{Controller}}
Der Controller der für die Steuerung der Software verantworlicch ist Enthalten sind:
\begin{itemize}
\item {} 
Verwaltung der View

\item {} 
Verwaltung des Dialog Fenster zum ändern, hinzufügen und löschen eines Knoten

\item {} 
Verwalten des Binärbaumes

\item {} 
Action Listener

\end{itemize}

\end{fulllineitems}



\paragraph{Constructors}
\label{\detokenize{com/linuxluigi/edu/Controller:constructors}}

\subparagraph{Controller}
\label{\detokenize{com/linuxluigi/edu/Controller:id1}}\index{Controller(View) (Java constructor)}

\begin{fulllineitems}
\phantomsection\label{\detokenize{com/linuxluigi/edu/Controller:com.linuxluigi.edu.Controller.Controller(View)}}\pysiglinewithargsret{public \sphinxbfcode{Controller}}{{\hyperref[\detokenize{com/linuxluigi/edu/view/View:com.linuxluigi.edu.view.View}]{\sphinxcrossref{View}}}\sphinxstyleemphasis{ view}}{}
Konstruktor des Controllers
\begin{itemize}
\item {} \begin{enumerate}
\item {} 
Erstellt ein Demo Binärbaum

\end{enumerate}

\item {} \begin{enumerate}
\setcounter{enumi}{1}
\item {} 
Übergibt die Binärbaum informationen der View

\end{enumerate}

\item {} \begin{enumerate}
\setcounter{enumi}{2}
\item {} 
Fügt die Actionen Listener für jeden Button hinzu

\end{enumerate}

\end{itemize}
\begin{quote}\begin{description}
\item[{Parameter}] \leavevmode\begin{itemize}
\item {} 
\sphinxstyleliteralstrong{view} -- Die View für das haupt Fenster

\end{itemize}

\end{description}\end{quote}

\end{fulllineitems}



\paragraph{Methods}
\label{\detokenize{com/linuxluigi/edu/Controller:methods}}

\subparagraph{updateView}
\label{\detokenize{com/linuxluigi/edu/Controller:updateview}}\index{updateView() (Java method)}

\begin{fulllineitems}
\phantomsection\label{\detokenize{com/linuxluigi/edu/Controller:com.linuxluigi.edu.Controller.updateView()}}\pysiglinewithargsret{ void \sphinxbfcode{updateView}}{}{}
Führt ein update der View aus
\begin{itemize}
\item {} \begin{enumerate}
\item {} 
Binärbaum der View übergeben

\end{enumerate}

\item {} \begin{enumerate}
\setcounter{enumi}{1}
\item {} 
Actionlistener einfügen

\end{enumerate}

\end{itemize}

\end{fulllineitems}



\subparagraph{updateViewInNewWindow}
\label{\detokenize{com/linuxluigi/edu/Controller:updateviewinnewwindow}}\index{updateViewInNewWindow() (Java method)}

\begin{fulllineitems}
\phantomsection\label{\detokenize{com/linuxluigi/edu/Controller:com.linuxluigi.edu.Controller.updateViewInNewWindow()}}\pysiglinewithargsret{ void \sphinxbfcode{updateViewInNewWindow}}{}{}
Führt ein update der View in ein neues Fenster aus und schließt das vorherige
\begin{itemize}
\item {} \begin{enumerate}
\item {} 
Position der alten View sichern

\end{enumerate}

\item {} \begin{enumerate}
\setcounter{enumi}{1}
\item {} 
Alte View unsichbar schalten

\end{enumerate}

\item {} \begin{enumerate}
\setcounter{enumi}{2}
\item {} 
Neue View an gleicher Stelle und gleicher Dimension der alten View erstellen

\end{enumerate}

\item {} \begin{enumerate}
\setcounter{enumi}{3}
\item {} 
Binärbaum der View übergeben

\end{enumerate}

\item {} \begin{enumerate}
\setcounter{enumi}{4}
\item {} 
Actionlistener einfügen

\end{enumerate}

\end{itemize}

\end{fulllineitems}



\subsubsection{Controller.DialogAddListener}
\label{\detokenize{com/linuxluigi/edu/Controller-DialogAddListener::doc}}\label{\detokenize{com/linuxluigi/edu/Controller-DialogAddListener:controller-dialogaddlistener}}\index{DialogAddListener (Java class)}

\begin{fulllineitems}
\phantomsection\label{\detokenize{com/linuxluigi/edu/Controller-DialogAddListener:com.linuxluigi.edu.Controller.DialogAddListener}}\pysigline{ class \sphinxbfcode{DialogAddListener} implements \href{http://docs.oracle.com/javase/8/docs/api/java/awt/event/ActionListener.html}{ActionListener}}
Actionelistener für Dialog Window:
\begin{itemize}
\item {} 
Hinzufügen von neuen Knoten

\end{itemize}

\end{fulllineitems}



\paragraph{Methods}
\label{\detokenize{com/linuxluigi/edu/Controller-DialogAddListener:methods}}

\subparagraph{actionPerformed}
\label{\detokenize{com/linuxluigi/edu/Controller-DialogAddListener:actionperformed}}\index{actionPerformed(ActionEvent) (Java method)}

\begin{fulllineitems}
\phantomsection\label{\detokenize{com/linuxluigi/edu/Controller-DialogAddListener:com.linuxluigi.edu.Controller.DialogAddListener.actionPerformed(ActionEvent)}}\pysiglinewithargsret{public void \sphinxbfcode{actionPerformed}}{\href{http://docs.oracle.com/javase/8/docs/api/java/awt/event/ActionEvent.html}{ActionEvent}\sphinxstyleemphasis{ arg0}}{}
\end{fulllineitems}



\subsubsection{Controller.DialogRemoveListener}
\label{\detokenize{com/linuxluigi/edu/Controller-DialogRemoveListener::doc}}\label{\detokenize{com/linuxluigi/edu/Controller-DialogRemoveListener:controller-dialogremovelistener}}\index{DialogRemoveListener (Java class)}

\begin{fulllineitems}
\phantomsection\label{\detokenize{com/linuxluigi/edu/Controller-DialogRemoveListener:com.linuxluigi.edu.Controller.DialogRemoveListener}}\pysigline{ class \sphinxbfcode{DialogRemoveListener} implements \href{http://docs.oracle.com/javase/8/docs/api/java/awt/event/ActionListener.html}{ActionListener}}
Actionelistener für Dialog Window:
\begin{itemize}
\item {} 
löschen eines Knotens

\end{itemize}

\end{fulllineitems}



\paragraph{Methods}
\label{\detokenize{com/linuxluigi/edu/Controller-DialogRemoveListener:methods}}

\subparagraph{actionPerformed}
\label{\detokenize{com/linuxluigi/edu/Controller-DialogRemoveListener:actionperformed}}\index{actionPerformed(ActionEvent) (Java method)}

\begin{fulllineitems}
\phantomsection\label{\detokenize{com/linuxluigi/edu/Controller-DialogRemoveListener:com.linuxluigi.edu.Controller.DialogRemoveListener.actionPerformed(ActionEvent)}}\pysiglinewithargsret{public void \sphinxbfcode{actionPerformed}}{\href{http://docs.oracle.com/javase/8/docs/api/java/awt/event/ActionEvent.html}{ActionEvent}\sphinxstyleemphasis{ arg0}}{}
\end{fulllineitems}



\subsubsection{Controller.DialogRenameListener}
\label{\detokenize{com/linuxluigi/edu/Controller-DialogRenameListener::doc}}\label{\detokenize{com/linuxluigi/edu/Controller-DialogRenameListener:controller-dialogrenamelistener}}\index{DialogRenameListener (Java class)}

\begin{fulllineitems}
\phantomsection\label{\detokenize{com/linuxluigi/edu/Controller-DialogRenameListener:com.linuxluigi.edu.Controller.DialogRenameListener}}\pysigline{ class \sphinxbfcode{DialogRenameListener} implements \href{http://docs.oracle.com/javase/8/docs/api/java/awt/event/ActionListener.html}{ActionListener}}
Actionelistener für Dialog Window:
\begin{itemize}
\item {} 
ändern eines vorhandenen Knoten

\end{itemize}

\end{fulllineitems}



\paragraph{Methods}
\label{\detokenize{com/linuxluigi/edu/Controller-DialogRenameListener:methods}}

\subparagraph{actionPerformed}
\label{\detokenize{com/linuxluigi/edu/Controller-DialogRenameListener:actionperformed}}\index{actionPerformed(ActionEvent) (Java method)}

\begin{fulllineitems}
\phantomsection\label{\detokenize{com/linuxluigi/edu/Controller-DialogRenameListener:com.linuxluigi.edu.Controller.DialogRenameListener.actionPerformed(ActionEvent)}}\pysiglinewithargsret{public void \sphinxbfcode{actionPerformed}}{\href{http://docs.oracle.com/javase/8/docs/api/java/awt/event/ActionEvent.html}{ActionEvent}\sphinxstyleemphasis{ arg0}}{}
\end{fulllineitems}



\subsubsection{Controller.MenuExitListener}
\label{\detokenize{com/linuxluigi/edu/Controller-MenuExitListener::doc}}\label{\detokenize{com/linuxluigi/edu/Controller-MenuExitListener:controller-menuexitlistener}}\index{MenuExitListener (Java class)}

\begin{fulllineitems}
\phantomsection\label{\detokenize{com/linuxluigi/edu/Controller-MenuExitListener:com.linuxluigi.edu.Controller.MenuExitListener}}\pysigline{ class \sphinxbfcode{MenuExitListener} implements \href{http://docs.oracle.com/javase/8/docs/api/java/awt/event/ActionListener.html}{ActionListener}}
Actionelistener für Menü Button: Exit

\end{fulllineitems}



\paragraph{Methods}
\label{\detokenize{com/linuxluigi/edu/Controller-MenuExitListener:methods}}

\subparagraph{actionPerformed}
\label{\detokenize{com/linuxluigi/edu/Controller-MenuExitListener:actionperformed}}\index{actionPerformed(ActionEvent) (Java method)}

\begin{fulllineitems}
\phantomsection\label{\detokenize{com/linuxluigi/edu/Controller-MenuExitListener:com.linuxluigi.edu.Controller.MenuExitListener.actionPerformed(ActionEvent)}}\pysiglinewithargsret{public void \sphinxbfcode{actionPerformed}}{\href{http://docs.oracle.com/javase/8/docs/api/java/awt/event/ActionEvent.html}{ActionEvent}\sphinxstyleemphasis{ arg0}}{}
\end{fulllineitems}



\subsubsection{Controller.MenuLoadListener}
\label{\detokenize{com/linuxluigi/edu/Controller-MenuLoadListener::doc}}\label{\detokenize{com/linuxluigi/edu/Controller-MenuLoadListener:controller-menuloadlistener}}\index{MenuLoadListener (Java class)}

\begin{fulllineitems}
\phantomsection\label{\detokenize{com/linuxluigi/edu/Controller-MenuLoadListener:com.linuxluigi.edu.Controller.MenuLoadListener}}\pysigline{ class \sphinxbfcode{MenuLoadListener} implements \href{http://docs.oracle.com/javase/8/docs/api/java/awt/event/ActionListener.html}{ActionListener}}
Actionelistener für Menü Button: Binärbaum aus Json Datei laden

\end{fulllineitems}



\paragraph{Methods}
\label{\detokenize{com/linuxluigi/edu/Controller-MenuLoadListener:methods}}

\subparagraph{actionPerformed}
\label{\detokenize{com/linuxluigi/edu/Controller-MenuLoadListener:actionperformed}}\index{actionPerformed(ActionEvent) (Java method)}

\begin{fulllineitems}
\phantomsection\label{\detokenize{com/linuxluigi/edu/Controller-MenuLoadListener:com.linuxluigi.edu.Controller.MenuLoadListener.actionPerformed(ActionEvent)}}\pysiglinewithargsret{public void \sphinxbfcode{actionPerformed}}{\href{http://docs.oracle.com/javase/8/docs/api/java/awt/event/ActionEvent.html}{ActionEvent}\sphinxstyleemphasis{ arg0}}{}
\end{fulllineitems}



\subsubsection{Controller.MenuNewListener}
\label{\detokenize{com/linuxluigi/edu/Controller-MenuNewListener::doc}}\label{\detokenize{com/linuxluigi/edu/Controller-MenuNewListener:controller-menunewlistener}}\index{MenuNewListener (Java class)}

\begin{fulllineitems}
\phantomsection\label{\detokenize{com/linuxluigi/edu/Controller-MenuNewListener:com.linuxluigi.edu.Controller.MenuNewListener}}\pysigline{ class \sphinxbfcode{MenuNewListener} implements \href{http://docs.oracle.com/javase/8/docs/api/java/awt/event/ActionListener.html}{ActionListener}}
Actionelistener für Menü Button: neuen Baum anlegen

\end{fulllineitems}



\paragraph{Methods}
\label{\detokenize{com/linuxluigi/edu/Controller-MenuNewListener:methods}}

\subparagraph{actionPerformed}
\label{\detokenize{com/linuxluigi/edu/Controller-MenuNewListener:actionperformed}}\index{actionPerformed(ActionEvent) (Java method)}

\begin{fulllineitems}
\phantomsection\label{\detokenize{com/linuxluigi/edu/Controller-MenuNewListener:com.linuxluigi.edu.Controller.MenuNewListener.actionPerformed(ActionEvent)}}\pysiglinewithargsret{public void \sphinxbfcode{actionPerformed}}{\href{http://docs.oracle.com/javase/8/docs/api/java/awt/event/ActionEvent.html}{ActionEvent}\sphinxstyleemphasis{ arg0}}{}
\end{fulllineitems}



\subsubsection{Controller.MenuSaveListener}
\label{\detokenize{com/linuxluigi/edu/Controller-MenuSaveListener::doc}}\label{\detokenize{com/linuxluigi/edu/Controller-MenuSaveListener:controller-menusavelistener}}\index{MenuSaveListener (Java class)}

\begin{fulllineitems}
\phantomsection\label{\detokenize{com/linuxluigi/edu/Controller-MenuSaveListener:com.linuxluigi.edu.Controller.MenuSaveListener}}\pysigline{ class \sphinxbfcode{MenuSaveListener} implements \href{http://docs.oracle.com/javase/8/docs/api/java/awt/event/ActionListener.html}{ActionListener}}
Actionelistener für Menü Button: Binärbaum in Json Datei speicher

\end{fulllineitems}



\paragraph{Methods}
\label{\detokenize{com/linuxluigi/edu/Controller-MenuSaveListener:methods}}

\subparagraph{actionPerformed}
\label{\detokenize{com/linuxluigi/edu/Controller-MenuSaveListener:actionperformed}}\index{actionPerformed(ActionEvent) (Java method)}

\begin{fulllineitems}
\phantomsection\label{\detokenize{com/linuxluigi/edu/Controller-MenuSaveListener:com.linuxluigi.edu.Controller.MenuSaveListener.actionPerformed(ActionEvent)}}\pysiglinewithargsret{public void \sphinxbfcode{actionPerformed}}{\href{http://docs.oracle.com/javase/8/docs/api/java/awt/event/ActionEvent.html}{ActionEvent}\sphinxstyleemphasis{ arg0}}{}
\end{fulllineitems}



\subsubsection{Controller.MenuSortAcsListener}
\label{\detokenize{com/linuxluigi/edu/Controller-MenuSortAcsListener::doc}}\label{\detokenize{com/linuxluigi/edu/Controller-MenuSortAcsListener:controller-menusortacslistener}}\index{MenuSortAcsListener (Java class)}

\begin{fulllineitems}
\phantomsection\label{\detokenize{com/linuxluigi/edu/Controller-MenuSortAcsListener:com.linuxluigi.edu.Controller.MenuSortAcsListener}}\pysigline{ class \sphinxbfcode{MenuSortAcsListener} implements \href{http://docs.oracle.com/javase/8/docs/api/java/awt/event/ActionListener.html}{ActionListener}}
Actionelistener für Menü Button: Binärbaum nach ACS sortieren

\end{fulllineitems}



\paragraph{Methods}
\label{\detokenize{com/linuxluigi/edu/Controller-MenuSortAcsListener:methods}}

\subparagraph{actionPerformed}
\label{\detokenize{com/linuxluigi/edu/Controller-MenuSortAcsListener:actionperformed}}\index{actionPerformed(ActionEvent) (Java method)}

\begin{fulllineitems}
\phantomsection\label{\detokenize{com/linuxluigi/edu/Controller-MenuSortAcsListener:com.linuxluigi.edu.Controller.MenuSortAcsListener.actionPerformed(ActionEvent)}}\pysiglinewithargsret{public void \sphinxbfcode{actionPerformed}}{\href{http://docs.oracle.com/javase/8/docs/api/java/awt/event/ActionEvent.html}{ActionEvent}\sphinxstyleemphasis{ arg0}}{}
\end{fulllineitems}



\subsubsection{Controller.MenuSortDecsListener}
\label{\detokenize{com/linuxluigi/edu/Controller-MenuSortDecsListener::doc}}\label{\detokenize{com/linuxluigi/edu/Controller-MenuSortDecsListener:controller-menusortdecslistener}}\index{MenuSortDecsListener (Java class)}

\begin{fulllineitems}
\phantomsection\label{\detokenize{com/linuxluigi/edu/Controller-MenuSortDecsListener:com.linuxluigi.edu.Controller.MenuSortDecsListener}}\pysigline{ class \sphinxbfcode{MenuSortDecsListener} implements \href{http://docs.oracle.com/javase/8/docs/api/java/awt/event/ActionListener.html}{ActionListener}}
Actionelistener für Menü Button: Binärbaum nach DECS sortieren

\end{fulllineitems}



\paragraph{Methods}
\label{\detokenize{com/linuxluigi/edu/Controller-MenuSortDecsListener:methods}}

\subparagraph{actionPerformed}
\label{\detokenize{com/linuxluigi/edu/Controller-MenuSortDecsListener:actionperformed}}\index{actionPerformed(ActionEvent) (Java method)}

\begin{fulllineitems}
\phantomsection\label{\detokenize{com/linuxluigi/edu/Controller-MenuSortDecsListener:com.linuxluigi.edu.Controller.MenuSortDecsListener.actionPerformed(ActionEvent)}}\pysiglinewithargsret{public void \sphinxbfcode{actionPerformed}}{\href{http://docs.oracle.com/javase/8/docs/api/java/awt/event/ActionEvent.html}{ActionEvent}\sphinxstyleemphasis{ arg0}}{}
\end{fulllineitems}



\subsubsection{Controller.NodeListener}
\label{\detokenize{com/linuxluigi/edu/Controller-NodeListener::doc}}\label{\detokenize{com/linuxluigi/edu/Controller-NodeListener:controller-nodelistener}}\index{NodeListener (Java class)}

\begin{fulllineitems}
\phantomsection\label{\detokenize{com/linuxluigi/edu/Controller-NodeListener:com.linuxluigi.edu.Controller.NodeListener}}\pysigline{ class \sphinxbfcode{NodeListener} implements \href{http://docs.oracle.com/javase/8/docs/api/java/awt/event/ActionListener.html}{ActionListener}}
Actionelistener für alle Binärbaum Knoten

\end{fulllineitems}



\paragraph{Methods}
\label{\detokenize{com/linuxluigi/edu/Controller-NodeListener:methods}}

\subparagraph{actionPerformed}
\label{\detokenize{com/linuxluigi/edu/Controller-NodeListener:actionperformed}}\index{actionPerformed(ActionEvent) (Java method)}

\begin{fulllineitems}
\phantomsection\label{\detokenize{com/linuxluigi/edu/Controller-NodeListener:com.linuxluigi.edu.Controller.NodeListener.actionPerformed(ActionEvent)}}\pysiglinewithargsret{public void \sphinxbfcode{actionPerformed}}{\href{http://docs.oracle.com/javase/8/docs/api/java/awt/event/ActionEvent.html}{ActionEvent}\sphinxstyleemphasis{ arg0}}{}
\end{fulllineitems}



\subsubsection{Load}
\label{\detokenize{com/linuxluigi/edu/Load::doc}}\label{\detokenize{com/linuxluigi/edu/Load:load}}\index{Load (Java class)}

\begin{fulllineitems}
\phantomsection\label{\detokenize{com/linuxluigi/edu/Load:com.linuxluigi.edu.Load}}\pysigline{public class \sphinxbfcode{Load}}
Lädt eine Json Datei und wandelt den Inhalt in ein Binärbaum um

\end{fulllineitems}



\paragraph{Constructors}
\label{\detokenize{com/linuxluigi/edu/Load:constructors}}

\subparagraph{Load}
\label{\detokenize{com/linuxluigi/edu/Load:id1}}\index{Load(File) (Java constructor)}

\begin{fulllineitems}
\phantomsection\label{\detokenize{com/linuxluigi/edu/Load:com.linuxluigi.edu.Load.Load(File)}}\pysiglinewithargsret{public \sphinxbfcode{Load}}{\href{http://docs.oracle.com/javase/8/docs/api/java/io/File.html}{File}\sphinxstyleemphasis{ file}}{}
Lädt eine Json Datei und wandelt den Inhalt in ein Binärbaum um
\begin{quote}\begin{description}
\item[{Parameter}] \leavevmode\begin{itemize}
\item {} 
\sphinxstyleliteralstrong{file} -- Datei die geladen werden soll

\end{itemize}

\end{description}\end{quote}

\end{fulllineitems}



\paragraph{Methods}
\label{\detokenize{com/linuxluigi/edu/Load:methods}}

\subparagraph{getBinaryListArray}
\label{\detokenize{com/linuxluigi/edu/Load:getbinarylistarray}}\index{getBinaryListArray() (Java method)}

\begin{fulllineitems}
\phantomsection\label{\detokenize{com/linuxluigi/edu/Load:com.linuxluigi.edu.Load.getBinaryListArray()}}\pysiglinewithargsret{public \href{http://docs.oracle.com/javase/8/docs/api/java/lang/String.html}{String}{[}{]}{[}{]} \sphinxbfcode{getBinaryListArray}}{}{}
\end{fulllineitems}



\subsubsection{Main}
\label{\detokenize{com/linuxluigi/edu/Main::doc}}\label{\detokenize{com/linuxluigi/edu/Main:main}}\index{Main (Java class)}

\begin{fulllineitems}
\phantomsection\label{\detokenize{com/linuxluigi/edu/Main:com.linuxluigi.edu.Main}}\pysigline{public class \sphinxbfcode{Main}}
Main Klasse zum starten des Programmes. Es wird eine erste View angelegt und dem Controller übergeben. Die View wird mit eine passende Größe für den ersten automatisch generierten Binär Baum erstellt.

\end{fulllineitems}



\paragraph{Methods}
\label{\detokenize{com/linuxluigi/edu/Main:methods}}

\subparagraph{main}
\label{\detokenize{com/linuxluigi/edu/Main:id1}}\index{main(String{[}{]}) (Java method)}

\begin{fulllineitems}
\phantomsection\label{\detokenize{com/linuxluigi/edu/Main:com.linuxluigi.edu.Main.main(String__)}}\pysiglinewithargsret{public static void \sphinxbfcode{main}}{\href{http://docs.oracle.com/javase/8/docs/api/java/lang/String.html}{String}{[}{]}\sphinxstyleemphasis{ args}}{}
Main Methode
\begin{quote}\begin{description}
\item[{Parameter}] \leavevmode\begin{itemize}
\item {} 
\sphinxstyleliteralstrong{args} -- 

\end{itemize}

\end{description}\end{quote}

\end{fulllineitems}



\subsubsection{Save}
\label{\detokenize{com/linuxluigi/edu/Save::doc}}\label{\detokenize{com/linuxluigi/edu/Save:save}}\index{Save (Java class)}

\begin{fulllineitems}
\phantomsection\label{\detokenize{com/linuxluigi/edu/Save:com.linuxluigi.edu.Save}}\pysigline{public class \sphinxbfcode{Save}}
Speichert den Binärbaum in eine Json Datei

\end{fulllineitems}



\paragraph{Constructors}
\label{\detokenize{com/linuxluigi/edu/Save:constructors}}

\subparagraph{Save}
\label{\detokenize{com/linuxluigi/edu/Save:id1}}\index{Save(File, Listlabel) (Java constructor)}

\begin{fulllineitems}
\phantomsection\label{\detokenize{com/linuxluigi/edu/Save:com.linuxluigi.edu.Save.Save(File, Listlabel)}}\pysiglinewithargsret{public \sphinxbfcode{Save}}{\href{http://docs.oracle.com/javase/8/docs/api/java/io/File.html}{File}\sphinxstyleemphasis{ file}, {\hyperref[\detokenize{com/linuxluigi/edu/list/Listlabel:com.linuxluigi.edu.list.Listlabel}]{\sphinxcrossref{Listlabel}}}\textless{}{\hyperref[\detokenize{com/linuxluigi/edu/data/NodeData:com.linuxluigi.edu.data.NodeData}]{\sphinxcrossref{NodeData}}}\textgreater{}\sphinxstyleemphasis{ nodeList}}{}
Speichert den Binärbaum in eine Json Datei Dabei wird der Binärbaum in 2 Dimensionalen String Array konvertiert, um speicherplatz effektiv zu sichern.
\begin{quote}\begin{description}
\item[{Parameter}] \leavevmode\begin{itemize}
\item {} 
\sphinxstyleliteralstrong{file} -- zu speichernde Json Datei

\item {} 
\sphinxstyleliteralstrong{nodeList} -- den zu Speicherenden Binärbaum

\end{itemize}

\end{description}\end{quote}

\end{fulllineitems}



\subsection{com.linuxluigi.edu.data}
\label{\detokenize{com/linuxluigi/edu/data/package-index:com-linuxluigi-edu-data}}\label{\detokenize{com/linuxluigi/edu/data/package-index::doc}}\label{\detokenize{com/linuxluigi/edu/data/package-index:package-com.linuxluigi.edu.data}}\index{com.linuxluigi.edu.data (package)}

\subsubsection{DrawnLines}
\label{\detokenize{com/linuxluigi/edu/data/DrawnLines::doc}}\label{\detokenize{com/linuxluigi/edu/data/DrawnLines:drawnlines}}\index{DrawnLines (Java class)}

\begin{fulllineitems}
\phantomsection\label{\detokenize{com/linuxluigi/edu/data/DrawnLines:com.linuxluigi.edu.data.DrawnLines}}\pysigline{public class \sphinxbfcode{DrawnLines}}
Daten für das Zeichnen von Linien zwischen den Knoten

\end{fulllineitems}



\paragraph{Constructors}
\label{\detokenize{com/linuxluigi/edu/data/DrawnLines:constructors}}

\subparagraph{DrawnLines}
\label{\detokenize{com/linuxluigi/edu/data/DrawnLines:id1}}\index{DrawnLines(int, int, int, int) (Java constructor)}

\begin{fulllineitems}
\phantomsection\label{\detokenize{com/linuxluigi/edu/data/DrawnLines:com.linuxluigi.edu.data.DrawnLines.DrawnLines(int, int, int, int)}}\pysiglinewithargsret{public \sphinxbfcode{DrawnLines}}{int\sphinxstyleemphasis{ x1}, int\sphinxstyleemphasis{ x2}, int\sphinxstyleemphasis{ y1}, int\sphinxstyleemphasis{ y2}}{}
Konstruktor: Daten für das Zeichnen von Linien zwischen den Knoten
\begin{quote}\begin{description}
\item[{Parameter}] \leavevmode\begin{itemize}
\item {} 
\sphinxstyleliteralstrong{x1} -- Start auf X-Achse

\item {} 
\sphinxstyleliteralstrong{x2} -- Ende auf X-Achse

\item {} 
\sphinxstyleliteralstrong{y1} -- Start auf Y-Achse

\item {} 
\sphinxstyleliteralstrong{y2} -- Ende auf Y-Achse

\end{itemize}

\end{description}\end{quote}

\end{fulllineitems}



\paragraph{Methods}
\label{\detokenize{com/linuxluigi/edu/data/DrawnLines:methods}}

\subparagraph{getX1}
\label{\detokenize{com/linuxluigi/edu/data/DrawnLines:getx1}}\index{getX1() (Java method)}

\begin{fulllineitems}
\phantomsection\label{\detokenize{com/linuxluigi/edu/data/DrawnLines:com.linuxluigi.edu.data.DrawnLines.getX1()}}\pysiglinewithargsret{public int \sphinxbfcode{getX1}}{}{}
\end{fulllineitems}



\subparagraph{getX2}
\label{\detokenize{com/linuxluigi/edu/data/DrawnLines:getx2}}\index{getX2() (Java method)}

\begin{fulllineitems}
\phantomsection\label{\detokenize{com/linuxluigi/edu/data/DrawnLines:com.linuxluigi.edu.data.DrawnLines.getX2()}}\pysiglinewithargsret{public int \sphinxbfcode{getX2}}{}{}
\end{fulllineitems}



\subparagraph{getY1}
\label{\detokenize{com/linuxluigi/edu/data/DrawnLines:gety1}}\index{getY1() (Java method)}

\begin{fulllineitems}
\phantomsection\label{\detokenize{com/linuxluigi/edu/data/DrawnLines:com.linuxluigi.edu.data.DrawnLines.getY1()}}\pysiglinewithargsret{public int \sphinxbfcode{getY1}}{}{}
\end{fulllineitems}



\subparagraph{getY2}
\label{\detokenize{com/linuxluigi/edu/data/DrawnLines:gety2}}\index{getY2() (Java method)}

\begin{fulllineitems}
\phantomsection\label{\detokenize{com/linuxluigi/edu/data/DrawnLines:com.linuxluigi.edu.data.DrawnLines.getY2()}}\pysiglinewithargsret{public int \sphinxbfcode{getY2}}{}{}
\end{fulllineitems}



\subsubsection{NodeData}
\label{\detokenize{com/linuxluigi/edu/data/NodeData::doc}}\label{\detokenize{com/linuxluigi/edu/data/NodeData:nodedata}}\index{NodeData (Java class)}

\begin{fulllineitems}
\phantomsection\label{\detokenize{com/linuxluigi/edu/data/NodeData:com.linuxluigi.edu.data.NodeData}}\pysigline{public class \sphinxbfcode{NodeData}}
Daten Objekt für die Knoten Gespeichert wird ein String mit der Maximalen länge von 3

\end{fulllineitems}



\paragraph{Constructors}
\label{\detokenize{com/linuxluigi/edu/data/NodeData:constructors}}

\subparagraph{NodeData}
\label{\detokenize{com/linuxluigi/edu/data/NodeData:id1}}\index{NodeData(String) (Java constructor)}

\begin{fulllineitems}
\phantomsection\label{\detokenize{com/linuxluigi/edu/data/NodeData:com.linuxluigi.edu.data.NodeData.NodeData(String)}}\pysiglinewithargsret{public \sphinxbfcode{NodeData}}{\href{http://docs.oracle.com/javase/8/docs/api/java/lang/String.html}{String}\sphinxstyleemphasis{ content}}{}
Daten Objekt für die Knoten Gespeichert wird ein String mit der Maximalen länge von 3

\end{fulllineitems}



\paragraph{Methods}
\label{\detokenize{com/linuxluigi/edu/data/NodeData:methods}}

\subparagraph{getContent}
\label{\detokenize{com/linuxluigi/edu/data/NodeData:getcontent}}\index{getContent() (Java method)}

\begin{fulllineitems}
\phantomsection\label{\detokenize{com/linuxluigi/edu/data/NodeData:com.linuxluigi.edu.data.NodeData.getContent()}}\pysiglinewithargsret{public \href{http://docs.oracle.com/javase/8/docs/api/java/lang/String.html}{String} \sphinxbfcode{getContent}}{}{}
\end{fulllineitems}



\subparagraph{setContent}
\label{\detokenize{com/linuxluigi/edu/data/NodeData:setcontent}}\index{setContent(String) (Java method)}

\begin{fulllineitems}
\phantomsection\label{\detokenize{com/linuxluigi/edu/data/NodeData:com.linuxluigi.edu.data.NodeData.setContent(String)}}\pysiglinewithargsret{public void \sphinxbfcode{setContent}}{\href{http://docs.oracle.com/javase/8/docs/api/java/lang/String.html}{String}\sphinxstyleemphasis{ content}}{}
Speichern eines String mit der Maximalen länge von 3, wenn der String länger ist, wird er automatisch auf 3 gekürtzt.
\begin{quote}\begin{description}
\item[{Parameter}] \leavevmode\begin{itemize}
\item {} 
\sphinxstyleliteralstrong{content} -- String der gesichert werden soll.

\end{itemize}

\end{description}\end{quote}

\end{fulllineitems}



\subparagraph{toString}
\label{\detokenize{com/linuxluigi/edu/data/NodeData:tostring}}\index{toString() (Java method)}

\begin{fulllineitems}
\phantomsection\label{\detokenize{com/linuxluigi/edu/data/NodeData:com.linuxluigi.edu.data.NodeData.toString()}}\pysiglinewithargsret{public \href{http://docs.oracle.com/javase/8/docs/api/java/lang/String.html}{String} \sphinxbfcode{toString}}{}{}
\end{fulllineitems}



\subsubsection{ViewPosition}
\label{\detokenize{com/linuxluigi/edu/data/ViewPosition::doc}}\label{\detokenize{com/linuxluigi/edu/data/ViewPosition:viewposition}}\index{ViewPosition (Java class)}

\begin{fulllineitems}
\phantomsection\label{\detokenize{com/linuxluigi/edu/data/ViewPosition:com.linuxluigi.edu.data.ViewPosition}}\pysigline{public class \sphinxbfcode{ViewPosition}}
Object um die Position der Knoten zu sichern

\end{fulllineitems}



\paragraph{Constructors}
\label{\detokenize{com/linuxluigi/edu/data/ViewPosition:constructors}}

\subparagraph{ViewPosition}
\label{\detokenize{com/linuxluigi/edu/data/ViewPosition:id1}}\index{ViewPosition(int, int, int) (Java constructor)}

\begin{fulllineitems}
\phantomsection\label{\detokenize{com/linuxluigi/edu/data/ViewPosition:com.linuxluigi.edu.data.ViewPosition.ViewPosition(int, int, int)}}\pysiglinewithargsret{public \sphinxbfcode{ViewPosition}}{int\sphinxstyleemphasis{ x}, int\sphinxstyleemphasis{ y}, int\sphinxstyleemphasis{ IconSize}}{}
Konstruktor um die Position der Knoten zu sichern
\begin{quote}\begin{description}
\item[{Parameter}] \leavevmode\begin{itemize}
\item {} 
\sphinxstyleliteralstrong{x} -- Startwert auf der X-Achse

\item {} 
\sphinxstyleliteralstrong{y} -- Startwert auf der Y-Achse

\item {} 
\sphinxstyleliteralstrong{IconSize} -- Größe des verwendeten Icons

\end{itemize}

\end{description}\end{quote}

\end{fulllineitems}



\paragraph{Methods}
\label{\detokenize{com/linuxluigi/edu/data/ViewPosition:methods}}

\subparagraph{getIconSize}
\label{\detokenize{com/linuxluigi/edu/data/ViewPosition:geticonsize}}\index{getIconSize() (Java method)}

\begin{fulllineitems}
\phantomsection\label{\detokenize{com/linuxluigi/edu/data/ViewPosition:com.linuxluigi.edu.data.ViewPosition.getIconSize()}}\pysiglinewithargsret{public int \sphinxbfcode{getIconSize}}{}{}
Gibt die Pixel Größe des Icons zurück
\begin{quote}\begin{description}
\item[{Rückgabe}] \leavevmode
Pixel Größe des Icons als INT

\end{description}\end{quote}

\end{fulllineitems}



\subparagraph{getMiddleX}
\label{\detokenize{com/linuxluigi/edu/data/ViewPosition:getmiddlex}}\index{getMiddleX() (Java method)}

\begin{fulllineitems}
\phantomsection\label{\detokenize{com/linuxluigi/edu/data/ViewPosition:com.linuxluigi.edu.data.ViewPosition.getMiddleX()}}\pysiglinewithargsret{public int \sphinxbfcode{getMiddleX}}{}{}
Gibt die Mitte des Objektes auf der X-Achse zurück
\begin{quote}\begin{description}
\item[{Rückgabe}] \leavevmode
Mitte des Objektes auf der X-Achse

\end{description}\end{quote}

\end{fulllineitems}



\subparagraph{getX}
\label{\detokenize{com/linuxluigi/edu/data/ViewPosition:getx}}\index{getX() (Java method)}

\begin{fulllineitems}
\phantomsection\label{\detokenize{com/linuxluigi/edu/data/ViewPosition:com.linuxluigi.edu.data.ViewPosition.getX()}}\pysiglinewithargsret{public int \sphinxbfcode{getX}}{}{}
Gibt den Startwert der X-Achse zurück
\begin{quote}\begin{description}
\item[{Rückgabe}] \leavevmode
Startwert der X-Achse

\end{description}\end{quote}

\end{fulllineitems}



\subparagraph{getY}
\label{\detokenize{com/linuxluigi/edu/data/ViewPosition:gety}}\index{getY() (Java method)}

\begin{fulllineitems}
\phantomsection\label{\detokenize{com/linuxluigi/edu/data/ViewPosition:com.linuxluigi.edu.data.ViewPosition.getY()}}\pysiglinewithargsret{public int \sphinxbfcode{getY}}{}{}
Gibt den Startwert der Y-Achse zurück
\begin{quote}\begin{description}
\item[{Rückgabe}] \leavevmode
Startwert der Y-Achse

\end{description}\end{quote}

\end{fulllineitems}



\subparagraph{getY2}
\label{\detokenize{com/linuxluigi/edu/data/ViewPosition:gety2}}\index{getY2() (Java method)}

\begin{fulllineitems}
\phantomsection\label{\detokenize{com/linuxluigi/edu/data/ViewPosition:com.linuxluigi.edu.data.ViewPosition.getY2()}}\pysiglinewithargsret{public int \sphinxbfcode{getY2}}{}{}
Gibt den Endwert der Y-Achse zurück
\begin{quote}\begin{description}
\item[{Rückgabe}] \leavevmode
Endwert der Y-Achse

\end{description}\end{quote}

\end{fulllineitems}



\subsection{com.linuxluigi.edu.list}
\label{\detokenize{com/linuxluigi/edu/list/package-index::doc}}\label{\detokenize{com/linuxluigi/edu/list/package-index:package-com.linuxluigi.edu.list}}\label{\detokenize{com/linuxluigi/edu/list/package-index:com-linuxluigi-edu-list}}\index{com.linuxluigi.edu.list (package)}

\subsubsection{BinaryLinkedList}
\label{\detokenize{com/linuxluigi/edu/list/BinaryLinkedList::doc}}\label{\detokenize{com/linuxluigi/edu/list/BinaryLinkedList:binarylinkedlist}}\index{BinaryLinkedList (Java class)}

\begin{fulllineitems}
\phantomsection\label{\detokenize{com/linuxluigi/edu/list/BinaryLinkedList:com.linuxluigi.edu.list.BinaryLinkedList}}\pysigline{public class \sphinxbfcode{BinaryLinkedList}\textless{}T\textgreater{} implements {\hyperref[\detokenize{com/linuxluigi/edu/list/Listlabel:com.linuxluigi.edu.list.Listlabel}]{\sphinxcrossref{Listlabel}}}\textless{}T\textgreater{}}
Binärbaum Liste
\begin{quote}\begin{description}
\item[{Parameter}] \leavevmode\begin{itemize}
\item {} 
\sphinxstyleliteralstrong{\textless{}T\textgreater{}} -- 

\end{itemize}

\end{description}\end{quote}

\end{fulllineitems}



\paragraph{Methods}
\label{\detokenize{com/linuxluigi/edu/list/BinaryLinkedList:methods}}

\subparagraph{add}
\label{\detokenize{com/linuxluigi/edu/list/BinaryLinkedList:add}}\index{add(T) (Java method)}

\begin{fulllineitems}
\phantomsection\label{\detokenize{com/linuxluigi/edu/list/BinaryLinkedList:com.linuxluigi.edu.list.BinaryLinkedList.add(T)}}\pysiglinewithargsret{public void \sphinxbfcode{add}}{T\sphinxstyleemphasis{ data}}{}
Fügt ein neuen Knoten ein
\begin{quote}\begin{description}
\item[{Parameter}] \leavevmode\begin{itemize}
\item {} 
\sphinxstyleliteralstrong{data} -- Datenobject

\end{itemize}

\end{description}\end{quote}

\end{fulllineitems}



\subparagraph{add}
\label{\detokenize{com/linuxluigi/edu/list/BinaryLinkedList:id1}}\index{add(int, T) (Java method)}

\begin{fulllineitems}
\phantomsection\label{\detokenize{com/linuxluigi/edu/list/BinaryLinkedList:com.linuxluigi.edu.list.BinaryLinkedList.add(int, T)}}\pysiglinewithargsret{public void \sphinxbfcode{add}}{int\sphinxstyleemphasis{ index}, T\sphinxstyleemphasis{ data}}{}
Fügt ein neuen nach dem Knoten mit der ID index ein Knoten ein
\begin{quote}\begin{description}
\item[{Parameter}] \leavevmode\begin{itemize}
\item {} 
\sphinxstyleliteralstrong{index} -- Index des Knotens

\item {} 
\sphinxstyleliteralstrong{data} -- Datenobject

\end{itemize}

\end{description}\end{quote}

\end{fulllineitems}



\subparagraph{clearAll}
\label{\detokenize{com/linuxluigi/edu/list/BinaryLinkedList:clearall}}\index{clearAll() (Java method)}

\begin{fulllineitems}
\phantomsection\label{\detokenize{com/linuxluigi/edu/list/BinaryLinkedList:com.linuxluigi.edu.list.BinaryLinkedList.clearAll()}}\pysiglinewithargsret{public void \sphinxbfcode{clearAll}}{}{}
Löscht alle Knoten aus der Liste

\end{fulllineitems}



\subparagraph{get}
\label{\detokenize{com/linuxluigi/edu/list/BinaryLinkedList:get}}\index{get(int) (Java method)}

\begin{fulllineitems}
\phantomsection\label{\detokenize{com/linuxluigi/edu/list/BinaryLinkedList:com.linuxluigi.edu.list.BinaryLinkedList.get(int)}}\pysiglinewithargsret{public T \sphinxbfcode{get}}{int\sphinxstyleemphasis{ index}}{}
Gibt den Wert vom Knoten mit dem Index index
\begin{quote}\begin{description}
\item[{Parameter}] \leavevmode\begin{itemize}
\item {} 
\sphinxstyleliteralstrong{index} -- Index des Knotens

\end{itemize}

\item[{Rückgabe}] \leavevmode
Daten Objekt

\end{description}\end{quote}

\end{fulllineitems}



\subparagraph{getBinaryList}
\label{\detokenize{com/linuxluigi/edu/list/BinaryLinkedList:getbinarylist}}\index{getBinaryList() (Java method)}

\begin{fulllineitems}
\phantomsection\label{\detokenize{com/linuxluigi/edu/list/BinaryLinkedList:com.linuxluigi.edu.list.BinaryLinkedList.getBinaryList()}}\pysiglinewithargsret{public \href{http://docs.oracle.com/javase/8/docs/api/java/lang/String.html}{String}{[}{]}{[}{]} \sphinxbfcode{getBinaryList}}{}{}
Konvertiert ein Binärbaum in ein String{[}{]}{[}{]}
\begin{quote}\begin{description}
\item[{Rückgabe}] \leavevmode
konvertierter Binärbaum als String{[}{]}{[}{]}

\end{description}\end{quote}

\end{fulllineitems}



\subparagraph{getDrawnLines}
\label{\detokenize{com/linuxluigi/edu/list/BinaryLinkedList:getdrawnlines}}\index{getDrawnLines() (Java method)}

\begin{fulllineitems}
\phantomsection\label{\detokenize{com/linuxluigi/edu/list/BinaryLinkedList:com.linuxluigi.edu.list.BinaryLinkedList.getDrawnLines()}}\pysiglinewithargsret{public {\hyperref[\detokenize{com/linuxluigi/edu/list/Listlabel:com.linuxluigi.edu.list.Listlabel}]{\sphinxcrossref{Listlabel}}}\textless{}{\hyperref[\detokenize{com/linuxluigi/edu/data/DrawnLines:com.linuxluigi.edu.data.DrawnLines}]{\sphinxcrossref{DrawnLines}}}\textgreater{} \sphinxbfcode{getDrawnLines}}{}{}
Gibt die Liste DrawnLines zurück
\begin{quote}\begin{description}
\item[{Rückgabe}] \leavevmode
Liste DrawnLines

\end{description}\end{quote}

\end{fulllineitems}



\subparagraph{getHigh}
\label{\detokenize{com/linuxluigi/edu/list/BinaryLinkedList:gethigh}}\index{getHigh() (Java method)}

\begin{fulllineitems}
\phantomsection\label{\detokenize{com/linuxluigi/edu/list/BinaryLinkedList:com.linuxluigi.edu.list.BinaryLinkedList.getHigh()}}\pysiglinewithargsret{public int \sphinxbfcode{getHigh}}{}{}
Gibt die höhe des Baumes in Pixel zurück
\begin{quote}\begin{description}
\item[{Rückgabe}] \leavevmode
höhe des Baumes

\end{description}\end{quote}

\end{fulllineitems}



\subparagraph{getSize}
\label{\detokenize{com/linuxluigi/edu/list/BinaryLinkedList:getsize}}\index{getSize() (Java method)}

\begin{fulllineitems}
\phantomsection\label{\detokenize{com/linuxluigi/edu/list/BinaryLinkedList:com.linuxluigi.edu.list.BinaryLinkedList.getSize()}}\pysiglinewithargsret{public int \sphinxbfcode{getSize}}{}{}
Gibt die Anzahl von Knoten zurück
\begin{quote}\begin{description}
\item[{Rückgabe}] \leavevmode
Anzahl von Knoten

\end{description}\end{quote}

\end{fulllineitems}



\subparagraph{getViewPosition}
\label{\detokenize{com/linuxluigi/edu/list/BinaryLinkedList:getviewposition}}\index{getViewPosition(int) (Java method)}

\begin{fulllineitems}
\phantomsection\label{\detokenize{com/linuxluigi/edu/list/BinaryLinkedList:com.linuxluigi.edu.list.BinaryLinkedList.getViewPosition(int)}}\pysiglinewithargsret{public {\hyperref[\detokenize{com/linuxluigi/edu/data/ViewPosition:com.linuxluigi.edu.data.ViewPosition}]{\sphinxcrossref{ViewPosition}}} \sphinxbfcode{getViewPosition}}{int\sphinxstyleemphasis{ index}}{}
Gibt das ViewPosition objekt des Knoten mit dem Index index zurück
\begin{quote}\begin{description}
\item[{Parameter}] \leavevmode\begin{itemize}
\item {} 
\sphinxstyleliteralstrong{index} -- Index des Knotens

\end{itemize}

\item[{Rückgabe}] \leavevmode
ViewPosition objekt des Knoten mit dem Index

\end{description}\end{quote}

\end{fulllineitems}



\subparagraph{getWith}
\label{\detokenize{com/linuxluigi/edu/list/BinaryLinkedList:getwith}}\index{getWith() (Java method)}

\begin{fulllineitems}
\phantomsection\label{\detokenize{com/linuxluigi/edu/list/BinaryLinkedList:com.linuxluigi.edu.list.BinaryLinkedList.getWith()}}\pysiglinewithargsret{public int \sphinxbfcode{getWith}}{}{}
Gibt die breite des Baumes in Pixel zurück
\begin{quote}\begin{description}
\item[{Rückgabe}] \leavevmode
breite des Baumes

\end{description}\end{quote}

\end{fulllineitems}



\subparagraph{isEmpty}
\label{\detokenize{com/linuxluigi/edu/list/BinaryLinkedList:isempty}}\index{isEmpty() (Java method)}

\begin{fulllineitems}
\phantomsection\label{\detokenize{com/linuxluigi/edu/list/BinaryLinkedList:com.linuxluigi.edu.list.BinaryLinkedList.isEmpty()}}\pysiglinewithargsret{public boolean \sphinxbfcode{isEmpty}}{}{}
Gibt zurück ob die Liste leer ist
\begin{quote}\begin{description}
\item[{Rückgabe}] \leavevmode
True == Liste ohne Knoten False == in der Liste sind Knoten enthalten

\end{description}\end{quote}

\end{fulllineitems}



\subparagraph{remove}
\label{\detokenize{com/linuxluigi/edu/list/BinaryLinkedList:remove}}\index{remove(int) (Java method)}

\begin{fulllineitems}
\phantomsection\label{\detokenize{com/linuxluigi/edu/list/BinaryLinkedList:com.linuxluigi.edu.list.BinaryLinkedList.remove(int)}}\pysiglinewithargsret{public void \sphinxbfcode{remove}}{int\sphinxstyleemphasis{ index}}{}
Löscht ein Knoten mit dem Index index
\begin{quote}\begin{description}
\item[{Parameter}] \leavevmode\begin{itemize}
\item {} 
\sphinxstyleliteralstrong{index} -- Index des zu löschenden Knotens

\end{itemize}

\end{description}\end{quote}

\end{fulllineitems}



\subparagraph{set}
\label{\detokenize{com/linuxluigi/edu/list/BinaryLinkedList:set}}\index{set(int, T) (Java method)}

\begin{fulllineitems}
\phantomsection\label{\detokenize{com/linuxluigi/edu/list/BinaryLinkedList:com.linuxluigi.edu.list.BinaryLinkedList.set(int, T)}}\pysiglinewithargsret{public void \sphinxbfcode{set}}{int\sphinxstyleemphasis{ index}, T\sphinxstyleemphasis{ data}}{}
Sichert ein Objekt in den Knoten mit den Index index
\begin{quote}\begin{description}
\item[{Parameter}] \leavevmode\begin{itemize}
\item {} 
\sphinxstyleliteralstrong{index} -- Index des Knotens

\item {} 
\sphinxstyleliteralstrong{data} -- zu sicherendes Objekt

\end{itemize}

\end{description}\end{quote}

\end{fulllineitems}



\subparagraph{setBinaryTreeFromList}
\label{\detokenize{com/linuxluigi/edu/list/BinaryLinkedList:setbinarytreefromlist}}\index{setBinaryTreeFromList(String{[}{]}{[}{]}) (Java method)}

\begin{fulllineitems}
\phantomsection\label{\detokenize{com/linuxluigi/edu/list/BinaryLinkedList:com.linuxluigi.edu.list.BinaryLinkedList.setBinaryTreeFromList(String____)}}\pysiglinewithargsret{public void \sphinxbfcode{setBinaryTreeFromList}}{\href{http://docs.oracle.com/javase/8/docs/api/java/lang/String.html}{String}{[}{]}{[}{]}\sphinxstyleemphasis{ binaryTreeArray}}{}
Konvertiert ein String{[}{]}{[}{]} in ein Binärbaum
\begin{quote}\begin{description}
\item[{Parameter}] \leavevmode\begin{itemize}
\item {} 
\sphinxstyleliteralstrong{binaryTreeArray} -- zu konvertierendendes String{[}{]}{[}{]}

\end{itemize}

\end{description}\end{quote}

\end{fulllineitems}



\subparagraph{sort}
\label{\detokenize{com/linuxluigi/edu/list/BinaryLinkedList:sort}}\index{sort(OrderBy) (Java method)}

\begin{fulllineitems}
\phantomsection\label{\detokenize{com/linuxluigi/edu/list/BinaryLinkedList:com.linuxluigi.edu.list.BinaryLinkedList.sort(OrderBy)}}\pysiglinewithargsret{public void \sphinxbfcode{sort}}{{\hyperref[\detokenize{com/linuxluigi/edu/list/OrderBy:com.linuxluigi.edu.list.OrderBy}]{\sphinxcrossref{OrderBy}}}\sphinxstyleemphasis{ orderBy}}{}
Sortiert den Baum nach ASC order DECS
\begin{quote}\begin{description}
\item[{Parameter}] \leavevmode\begin{itemize}
\item {} 
\sphinxstyleliteralstrong{orderBy} -- OrderBy.ASC == Sortieren nach ASC OrderBy.DESC == Sortieren nach DESC

\end{itemize}

\end{description}\end{quote}

\end{fulllineitems}



\subsubsection{Listlabel}
\label{\detokenize{com/linuxluigi/edu/list/Listlabel:listlabel}}\label{\detokenize{com/linuxluigi/edu/list/Listlabel::doc}}\index{Listlabel (Java interface)}

\begin{fulllineitems}
\phantomsection\label{\detokenize{com/linuxluigi/edu/list/Listlabel:com.linuxluigi.edu.list.Listlabel}}\pysigline{public interface \sphinxbfcode{Listlabel}\textless{}T\textgreater{}}
\end{fulllineitems}



\paragraph{Methods}
\label{\detokenize{com/linuxluigi/edu/list/Listlabel:methods}}

\subparagraph{add}
\label{\detokenize{com/linuxluigi/edu/list/Listlabel:add}}\index{add(T) (Java method)}

\begin{fulllineitems}
\phantomsection\label{\detokenize{com/linuxluigi/edu/list/Listlabel:com.linuxluigi.edu.list.Listlabel.add(T)}}\pysiglinewithargsret{ void \sphinxbfcode{add}}{T\sphinxstyleemphasis{ data}}{}
Fügt ein neuen Knoten ein
\begin{quote}\begin{description}
\item[{Parameter}] \leavevmode\begin{itemize}
\item {} 
\sphinxstyleliteralstrong{data} -- Datenobject

\end{itemize}

\end{description}\end{quote}

\end{fulllineitems}



\subparagraph{add}
\label{\detokenize{com/linuxluigi/edu/list/Listlabel:id1}}\index{add(int, T) (Java method)}

\begin{fulllineitems}
\phantomsection\label{\detokenize{com/linuxluigi/edu/list/Listlabel:com.linuxluigi.edu.list.Listlabel.add(int, T)}}\pysiglinewithargsret{ void \sphinxbfcode{add}}{int\sphinxstyleemphasis{ index}, T\sphinxstyleemphasis{ data}}{}
Fügt ein neuen nach dem Knoten mit der ID index ein Knoten ein
\begin{quote}\begin{description}
\item[{Parameter}] \leavevmode\begin{itemize}
\item {} 
\sphinxstyleliteralstrong{index} -- Index des Knotens

\item {} 
\sphinxstyleliteralstrong{data} -- Datenobject

\end{itemize}

\end{description}\end{quote}

\end{fulllineitems}



\subparagraph{clearAll}
\label{\detokenize{com/linuxluigi/edu/list/Listlabel:clearall}}\index{clearAll() (Java method)}

\begin{fulllineitems}
\phantomsection\label{\detokenize{com/linuxluigi/edu/list/Listlabel:com.linuxluigi.edu.list.Listlabel.clearAll()}}\pysiglinewithargsret{ void \sphinxbfcode{clearAll}}{}{}
Löscht alle Knoten aus der Liste

\end{fulllineitems}



\subparagraph{get}
\label{\detokenize{com/linuxluigi/edu/list/Listlabel:get}}\index{get(int) (Java method)}

\begin{fulllineitems}
\phantomsection\label{\detokenize{com/linuxluigi/edu/list/Listlabel:com.linuxluigi.edu.list.Listlabel.get(int)}}\pysiglinewithargsret{ T \sphinxbfcode{get}}{int\sphinxstyleemphasis{ index}}{}
Gibt den Wert vom Knoten mit dem Index index
\begin{quote}\begin{description}
\item[{Parameter}] \leavevmode\begin{itemize}
\item {} 
\sphinxstyleliteralstrong{index} -- Index des Knotens

\end{itemize}

\item[{Rückgabe}] \leavevmode
Daten Objekt

\end{description}\end{quote}

\end{fulllineitems}



\subparagraph{getBinaryList}
\label{\detokenize{com/linuxluigi/edu/list/Listlabel:getbinarylist}}\index{getBinaryList() (Java method)}

\begin{fulllineitems}
\phantomsection\label{\detokenize{com/linuxluigi/edu/list/Listlabel:com.linuxluigi.edu.list.Listlabel.getBinaryList()}}\pysiglinewithargsret{ \href{http://docs.oracle.com/javase/8/docs/api/java/lang/String.html}{String}{[}{]}{[}{]} \sphinxbfcode{getBinaryList}}{}{}
Konvertiert ein Binärbaum in ein String{[}{]}{[}{]}
\begin{quote}\begin{description}
\item[{Rückgabe}] \leavevmode
konvertierter Binärbaum als String{[}{]}{[}{]}

\end{description}\end{quote}

\end{fulllineitems}



\subparagraph{getDrawnLines}
\label{\detokenize{com/linuxluigi/edu/list/Listlabel:getdrawnlines}}\index{getDrawnLines() (Java method)}

\begin{fulllineitems}
\phantomsection\label{\detokenize{com/linuxluigi/edu/list/Listlabel:com.linuxluigi.edu.list.Listlabel.getDrawnLines()}}\pysiglinewithargsret{ {\hyperref[\detokenize{com/linuxluigi/edu/list/Listlabel:com.linuxluigi.edu.list.Listlabel}]{\sphinxcrossref{Listlabel}}}\textless{}{\hyperref[\detokenize{com/linuxluigi/edu/data/DrawnLines:com.linuxluigi.edu.data.DrawnLines}]{\sphinxcrossref{DrawnLines}}}\textgreater{} \sphinxbfcode{getDrawnLines}}{}{}
Gibt die Liste DrawnLines zurück
\begin{quote}\begin{description}
\item[{Rückgabe}] \leavevmode
Liste DrawnLines

\end{description}\end{quote}

\end{fulllineitems}



\subparagraph{getHigh}
\label{\detokenize{com/linuxluigi/edu/list/Listlabel:gethigh}}\index{getHigh() (Java method)}

\begin{fulllineitems}
\phantomsection\label{\detokenize{com/linuxluigi/edu/list/Listlabel:com.linuxluigi.edu.list.Listlabel.getHigh()}}\pysiglinewithargsret{ int \sphinxbfcode{getHigh}}{}{}
Gibt die höhe des Baumes in Pixel zurück
\begin{quote}\begin{description}
\item[{Rückgabe}] \leavevmode
höhe des Baumes

\end{description}\end{quote}

\end{fulllineitems}



\subparagraph{getSize}
\label{\detokenize{com/linuxluigi/edu/list/Listlabel:getsize}}\index{getSize() (Java method)}

\begin{fulllineitems}
\phantomsection\label{\detokenize{com/linuxluigi/edu/list/Listlabel:com.linuxluigi.edu.list.Listlabel.getSize()}}\pysiglinewithargsret{ int \sphinxbfcode{getSize}}{}{}
Gibt die Anzahl von Knoten zurück
\begin{quote}\begin{description}
\item[{Rückgabe}] \leavevmode
Anzahl von Knoten

\end{description}\end{quote}

\end{fulllineitems}



\subparagraph{getViewPosition}
\label{\detokenize{com/linuxluigi/edu/list/Listlabel:getviewposition}}\index{getViewPosition(int) (Java method)}

\begin{fulllineitems}
\phantomsection\label{\detokenize{com/linuxluigi/edu/list/Listlabel:com.linuxluigi.edu.list.Listlabel.getViewPosition(int)}}\pysiglinewithargsret{ {\hyperref[\detokenize{com/linuxluigi/edu/data/ViewPosition:com.linuxluigi.edu.data.ViewPosition}]{\sphinxcrossref{ViewPosition}}} \sphinxbfcode{getViewPosition}}{int\sphinxstyleemphasis{ index}}{}
Gibt das ViewPosition objekt des Knoten mit dem Index index zurück
\begin{quote}\begin{description}
\item[{Parameter}] \leavevmode\begin{itemize}
\item {} 
\sphinxstyleliteralstrong{index} -- Index des Knotens

\end{itemize}

\item[{Rückgabe}] \leavevmode
ViewPosition objekt des Knoten mit dem Index

\end{description}\end{quote}

\end{fulllineitems}



\subparagraph{getWith}
\label{\detokenize{com/linuxluigi/edu/list/Listlabel:getwith}}\index{getWith() (Java method)}

\begin{fulllineitems}
\phantomsection\label{\detokenize{com/linuxluigi/edu/list/Listlabel:com.linuxluigi.edu.list.Listlabel.getWith()}}\pysiglinewithargsret{ int \sphinxbfcode{getWith}}{}{}
Gibt die breite des Baumes in Pixel zurück
\begin{quote}\begin{description}
\item[{Rückgabe}] \leavevmode
breite des Baumes

\end{description}\end{quote}

\end{fulllineitems}



\subparagraph{isEmpty}
\label{\detokenize{com/linuxluigi/edu/list/Listlabel:isempty}}\index{isEmpty() (Java method)}

\begin{fulllineitems}
\phantomsection\label{\detokenize{com/linuxluigi/edu/list/Listlabel:com.linuxluigi.edu.list.Listlabel.isEmpty()}}\pysiglinewithargsret{ boolean \sphinxbfcode{isEmpty}}{}{}
Gibt zurück ob die Liste leer ist
\begin{quote}\begin{description}
\item[{Rückgabe}] \leavevmode
True == Liste ohne Knoten False == in der Liste sind Knoten enthalten

\end{description}\end{quote}

\end{fulllineitems}



\subparagraph{remove}
\label{\detokenize{com/linuxluigi/edu/list/Listlabel:remove}}\index{remove(int) (Java method)}

\begin{fulllineitems}
\phantomsection\label{\detokenize{com/linuxluigi/edu/list/Listlabel:com.linuxluigi.edu.list.Listlabel.remove(int)}}\pysiglinewithargsret{ void \sphinxbfcode{remove}}{int\sphinxstyleemphasis{ index}}{}
Löscht ein Knoten mit dem Index index
\begin{quote}\begin{description}
\item[{Parameter}] \leavevmode\begin{itemize}
\item {} 
\sphinxstyleliteralstrong{index} -- Index des zu löschenden Knotens

\end{itemize}

\end{description}\end{quote}

\end{fulllineitems}



\subparagraph{set}
\label{\detokenize{com/linuxluigi/edu/list/Listlabel:set}}\index{set(int, T) (Java method)}

\begin{fulllineitems}
\phantomsection\label{\detokenize{com/linuxluigi/edu/list/Listlabel:com.linuxluigi.edu.list.Listlabel.set(int, T)}}\pysiglinewithargsret{ void \sphinxbfcode{set}}{int\sphinxstyleemphasis{ index}, T\sphinxstyleemphasis{ data}}{}
Sichert ein Objekt in den Knoten mit den Index index
\begin{quote}\begin{description}
\item[{Parameter}] \leavevmode\begin{itemize}
\item {} 
\sphinxstyleliteralstrong{index} -- Index des Knotens

\item {} 
\sphinxstyleliteralstrong{data} -- zu sicherendes Objekt

\end{itemize}

\end{description}\end{quote}

\end{fulllineitems}



\subparagraph{setBinaryTreeFromList}
\label{\detokenize{com/linuxluigi/edu/list/Listlabel:setbinarytreefromlist}}\index{setBinaryTreeFromList(String{[}{]}{[}{]}) (Java method)}

\begin{fulllineitems}
\phantomsection\label{\detokenize{com/linuxluigi/edu/list/Listlabel:com.linuxluigi.edu.list.Listlabel.setBinaryTreeFromList(String____)}}\pysiglinewithargsret{ void \sphinxbfcode{setBinaryTreeFromList}}{\href{http://docs.oracle.com/javase/8/docs/api/java/lang/String.html}{String}{[}{]}{[}{]}\sphinxstyleemphasis{ binaryTreeArray}}{}
Konvertiert ein String{[}{]}{[}{]} in ein Binärbaum
\begin{quote}\begin{description}
\item[{Parameter}] \leavevmode\begin{itemize}
\item {} 
\sphinxstyleliteralstrong{binaryTreeArray} -- zu konvertierendendes String{[}{]}{[}{]}

\end{itemize}

\end{description}\end{quote}

\end{fulllineitems}



\subparagraph{sort}
\label{\detokenize{com/linuxluigi/edu/list/Listlabel:sort}}\index{sort(OrderBy) (Java method)}

\begin{fulllineitems}
\phantomsection\label{\detokenize{com/linuxluigi/edu/list/Listlabel:com.linuxluigi.edu.list.Listlabel.sort(OrderBy)}}\pysiglinewithargsret{ void \sphinxbfcode{sort}}{{\hyperref[\detokenize{com/linuxluigi/edu/list/OrderBy:com.linuxluigi.edu.list.OrderBy}]{\sphinxcrossref{OrderBy}}}\sphinxstyleemphasis{ orderBy}}{}
Sortiert den Baum nach ASC order DECS
\begin{quote}\begin{description}
\item[{Parameter}] \leavevmode\begin{itemize}
\item {} 
\sphinxstyleliteralstrong{orderBy} -- OrderBy.ASC == Sortieren nach ASC OrderBy.DESC == Sortieren nach DESC

\end{itemize}

\end{description}\end{quote}

\end{fulllineitems}



\subsubsection{OrderBy}
\label{\detokenize{com/linuxluigi/edu/list/OrderBy::doc}}\label{\detokenize{com/linuxluigi/edu/list/OrderBy:orderby}}\index{OrderBy (Java enum)}

\begin{fulllineitems}
\phantomsection\label{\detokenize{com/linuxluigi/edu/list/OrderBy:com.linuxluigi.edu.list.OrderBy}}\pysigline{public enum \sphinxbfcode{OrderBy}}
\end{fulllineitems}



\paragraph{Enum Constants}
\label{\detokenize{com/linuxluigi/edu/list/OrderBy:enum-constants}}

\subparagraph{ASC}
\label{\detokenize{com/linuxluigi/edu/list/OrderBy:asc}}\index{ASC (Java field)}

\begin{fulllineitems}
\phantomsection\label{\detokenize{com/linuxluigi/edu/list/OrderBy:com.linuxluigi.edu.list.OrderBy.ASC}}\pysigline{public static final {\hyperref[\detokenize{com/linuxluigi/edu/list/OrderBy:com.linuxluigi.edu.list.OrderBy}]{\sphinxcrossref{OrderBy}}} \sphinxbfcode{ASC}}
\end{fulllineitems}



\subparagraph{DESC}
\label{\detokenize{com/linuxluigi/edu/list/OrderBy:desc}}\index{DESC (Java field)}

\begin{fulllineitems}
\phantomsection\label{\detokenize{com/linuxluigi/edu/list/OrderBy:com.linuxluigi.edu.list.OrderBy.DESC}}\pysigline{public static final {\hyperref[\detokenize{com/linuxluigi/edu/list/OrderBy:com.linuxluigi.edu.list.OrderBy}]{\sphinxcrossref{OrderBy}}} \sphinxbfcode{DESC}}
\end{fulllineitems}



\subsubsection{PrevNodeDirection}
\label{\detokenize{com/linuxluigi/edu/list/PrevNodeDirection::doc}}\label{\detokenize{com/linuxluigi/edu/list/PrevNodeDirection:prevnodedirection}}\index{PrevNodeDirection (Java enum)}

\begin{fulllineitems}
\phantomsection\label{\detokenize{com/linuxluigi/edu/list/PrevNodeDirection:com.linuxluigi.edu.list.PrevNodeDirection}}\pysigline{public enum \sphinxbfcode{PrevNodeDirection}}
Created by fubu on 07.02.17.

\end{fulllineitems}



\paragraph{Enum Constants}
\label{\detokenize{com/linuxluigi/edu/list/PrevNodeDirection:enum-constants}}

\subparagraph{DOWN\_LEFT}
\label{\detokenize{com/linuxluigi/edu/list/PrevNodeDirection:down-left}}\index{DOWN\_LEFT (Java field)}

\begin{fulllineitems}
\phantomsection\label{\detokenize{com/linuxluigi/edu/list/PrevNodeDirection:com.linuxluigi.edu.list.PrevNodeDirection.DOWN_LEFT}}\pysigline{public static final {\hyperref[\detokenize{com/linuxluigi/edu/list/PrevNodeDirection:com.linuxluigi.edu.list.PrevNodeDirection}]{\sphinxcrossref{PrevNodeDirection}}} \sphinxbfcode{DOWN\_LEFT}}
\end{fulllineitems}



\subparagraph{DOWN\_RIGHT}
\label{\detokenize{com/linuxluigi/edu/list/PrevNodeDirection:down-right}}\index{DOWN\_RIGHT (Java field)}

\begin{fulllineitems}
\phantomsection\label{\detokenize{com/linuxluigi/edu/list/PrevNodeDirection:com.linuxluigi.edu.list.PrevNodeDirection.DOWN_RIGHT}}\pysigline{public static final {\hyperref[\detokenize{com/linuxluigi/edu/list/PrevNodeDirection:com.linuxluigi.edu.list.PrevNodeDirection}]{\sphinxcrossref{PrevNodeDirection}}} \sphinxbfcode{DOWN\_RIGHT}}
\end{fulllineitems}



\subparagraph{NULL}
\label{\detokenize{com/linuxluigi/edu/list/PrevNodeDirection:null}}\index{NULL (Java field)}

\begin{fulllineitems}
\phantomsection\label{\detokenize{com/linuxluigi/edu/list/PrevNodeDirection:com.linuxluigi.edu.list.PrevNodeDirection.NULL}}\pysigline{public static final {\hyperref[\detokenize{com/linuxluigi/edu/list/PrevNodeDirection:com.linuxluigi.edu.list.PrevNodeDirection}]{\sphinxcrossref{PrevNodeDirection}}} \sphinxbfcode{NULL}}
\end{fulllineitems}



\subparagraph{UP\_LEFT}
\label{\detokenize{com/linuxluigi/edu/list/PrevNodeDirection:up-left}}\index{UP\_LEFT (Java field)}

\begin{fulllineitems}
\phantomsection\label{\detokenize{com/linuxluigi/edu/list/PrevNodeDirection:com.linuxluigi.edu.list.PrevNodeDirection.UP_LEFT}}\pysigline{public static final {\hyperref[\detokenize{com/linuxluigi/edu/list/PrevNodeDirection:com.linuxluigi.edu.list.PrevNodeDirection}]{\sphinxcrossref{PrevNodeDirection}}} \sphinxbfcode{UP\_LEFT}}
\end{fulllineitems}



\subparagraph{UP\_RIGHT}
\label{\detokenize{com/linuxluigi/edu/list/PrevNodeDirection:up-right}}\index{UP\_RIGHT (Java field)}

\begin{fulllineitems}
\phantomsection\label{\detokenize{com/linuxluigi/edu/list/PrevNodeDirection:com.linuxluigi.edu.list.PrevNodeDirection.UP_RIGHT}}\pysigline{public static final {\hyperref[\detokenize{com/linuxluigi/edu/list/PrevNodeDirection:com.linuxluigi.edu.list.PrevNodeDirection}]{\sphinxcrossref{PrevNodeDirection}}} \sphinxbfcode{UP\_RIGHT}}
\end{fulllineitems}



\subsubsection{SinglyLinkedList}
\label{\detokenize{com/linuxluigi/edu/list/SinglyLinkedList::doc}}\label{\detokenize{com/linuxluigi/edu/list/SinglyLinkedList:singlylinkedlist}}\index{SinglyLinkedList (Java class)}

\begin{fulllineitems}
\phantomsection\label{\detokenize{com/linuxluigi/edu/list/SinglyLinkedList:com.linuxluigi.edu.list.SinglyLinkedList}}\pysigline{public class \sphinxbfcode{SinglyLinkedList}\textless{}T\textgreater{} implements {\hyperref[\detokenize{com/linuxluigi/edu/list/Listlabel:com.linuxluigi.edu.list.Listlabel}]{\sphinxcrossref{Listlabel}}}\textless{}T\textgreater{}}
Simple Liste
\begin{quote}\begin{description}
\item[{Parameter}] \leavevmode\begin{itemize}
\item {} 
\sphinxstyleliteralstrong{\textless{}T\textgreater{}} -- 

\end{itemize}

\end{description}\end{quote}

\end{fulllineitems}



\paragraph{Methods}
\label{\detokenize{com/linuxluigi/edu/list/SinglyLinkedList:methods}}

\subparagraph{add}
\label{\detokenize{com/linuxluigi/edu/list/SinglyLinkedList:add}}\index{add(T) (Java method)}

\begin{fulllineitems}
\phantomsection\label{\detokenize{com/linuxluigi/edu/list/SinglyLinkedList:com.linuxluigi.edu.list.SinglyLinkedList.add(T)}}\pysiglinewithargsret{public void \sphinxbfcode{add}}{T\sphinxstyleemphasis{ data}}{}
Fügt ein neuen Knoten ein
\begin{quote}\begin{description}
\item[{Parameter}] \leavevmode\begin{itemize}
\item {} 
\sphinxstyleliteralstrong{data} -- Datenobject

\end{itemize}

\end{description}\end{quote}

\end{fulllineitems}



\subparagraph{add}
\label{\detokenize{com/linuxluigi/edu/list/SinglyLinkedList:id1}}\index{add(int, T) (Java method)}

\begin{fulllineitems}
\phantomsection\label{\detokenize{com/linuxluigi/edu/list/SinglyLinkedList:com.linuxluigi.edu.list.SinglyLinkedList.add(int, T)}}\pysiglinewithargsret{public void \sphinxbfcode{add}}{int\sphinxstyleemphasis{ index}, T\sphinxstyleemphasis{ data}}{}
Fügt ein neuen nach dem Knoten mit der ID index ein Knoten ein
\begin{quote}\begin{description}
\item[{Parameter}] \leavevmode\begin{itemize}
\item {} 
\sphinxstyleliteralstrong{index} -- Index des Knotens

\item {} 
\sphinxstyleliteralstrong{data} -- Datenobject

\end{itemize}

\end{description}\end{quote}

\end{fulllineitems}



\subparagraph{clearAll}
\label{\detokenize{com/linuxluigi/edu/list/SinglyLinkedList:clearall}}\index{clearAll() (Java method)}

\begin{fulllineitems}
\phantomsection\label{\detokenize{com/linuxluigi/edu/list/SinglyLinkedList:com.linuxluigi.edu.list.SinglyLinkedList.clearAll()}}\pysiglinewithargsret{public void \sphinxbfcode{clearAll}}{}{}
Löscht alle Knoten aus der Liste

\end{fulllineitems}



\subparagraph{get}
\label{\detokenize{com/linuxluigi/edu/list/SinglyLinkedList:get}}\index{get(int) (Java method)}

\begin{fulllineitems}
\phantomsection\label{\detokenize{com/linuxluigi/edu/list/SinglyLinkedList:com.linuxluigi.edu.list.SinglyLinkedList.get(int)}}\pysiglinewithargsret{public T \sphinxbfcode{get}}{int\sphinxstyleemphasis{ index}}{}
Gibt den Wert vom Knoten mit dem Index index
\begin{quote}\begin{description}
\item[{Parameter}] \leavevmode\begin{itemize}
\item {} 
\sphinxstyleliteralstrong{index} -- Index des Knotens

\end{itemize}

\item[{Rückgabe}] \leavevmode
Daten Objekt

\end{description}\end{quote}

\end{fulllineitems}



\subparagraph{getBinaryList}
\label{\detokenize{com/linuxluigi/edu/list/SinglyLinkedList:getbinarylist}}\index{getBinaryList() (Java method)}

\begin{fulllineitems}
\phantomsection\label{\detokenize{com/linuxluigi/edu/list/SinglyLinkedList:com.linuxluigi.edu.list.SinglyLinkedList.getBinaryList()}}\pysiglinewithargsret{public \href{http://docs.oracle.com/javase/8/docs/api/java/lang/String.html}{String}{[}{]}{[}{]} \sphinxbfcode{getBinaryList}}{}{}
Konvertiert ein Binärbaum in ein String{[}{]}{[}{]}
\begin{quote}\begin{description}
\item[{Rückgabe}] \leavevmode
konvertierter Binärbaum als String{[}{]}{[}{]}

\end{description}\end{quote}

\end{fulllineitems}



\subparagraph{getDrawnLines}
\label{\detokenize{com/linuxluigi/edu/list/SinglyLinkedList:getdrawnlines}}\index{getDrawnLines() (Java method)}

\begin{fulllineitems}
\phantomsection\label{\detokenize{com/linuxluigi/edu/list/SinglyLinkedList:com.linuxluigi.edu.list.SinglyLinkedList.getDrawnLines()}}\pysiglinewithargsret{public {\hyperref[\detokenize{com/linuxluigi/edu/list/Listlabel:com.linuxluigi.edu.list.Listlabel}]{\sphinxcrossref{Listlabel}}}\textless{}{\hyperref[\detokenize{com/linuxluigi/edu/data/DrawnLines:com.linuxluigi.edu.data.DrawnLines}]{\sphinxcrossref{DrawnLines}}}\textgreater{} \sphinxbfcode{getDrawnLines}}{}{}
Gibt die Liste DrawnLines zurück
\begin{quote}\begin{description}
\item[{Rückgabe}] \leavevmode
Liste DrawnLines

\end{description}\end{quote}

\end{fulllineitems}



\subparagraph{getHigh}
\label{\detokenize{com/linuxluigi/edu/list/SinglyLinkedList:gethigh}}\index{getHigh() (Java method)}

\begin{fulllineitems}
\phantomsection\label{\detokenize{com/linuxluigi/edu/list/SinglyLinkedList:com.linuxluigi.edu.list.SinglyLinkedList.getHigh()}}\pysiglinewithargsret{public int \sphinxbfcode{getHigh}}{}{}
Gibt die höhe des Baumes in Pixel zurück
\begin{quote}\begin{description}
\item[{Rückgabe}] \leavevmode
höhe des Baumes

\end{description}\end{quote}

\end{fulllineitems}



\subparagraph{getSize}
\label{\detokenize{com/linuxluigi/edu/list/SinglyLinkedList:getsize}}\index{getSize() (Java method)}

\begin{fulllineitems}
\phantomsection\label{\detokenize{com/linuxluigi/edu/list/SinglyLinkedList:com.linuxluigi.edu.list.SinglyLinkedList.getSize()}}\pysiglinewithargsret{public int \sphinxbfcode{getSize}}{}{}
Gibt die Anzahl von Knoten zurück
\begin{quote}\begin{description}
\item[{Rückgabe}] \leavevmode
Anzahl von Knoten

\end{description}\end{quote}

\end{fulllineitems}



\subparagraph{getViewPosition}
\label{\detokenize{com/linuxluigi/edu/list/SinglyLinkedList:getviewposition}}\index{getViewPosition(int) (Java method)}

\begin{fulllineitems}
\phantomsection\label{\detokenize{com/linuxluigi/edu/list/SinglyLinkedList:com.linuxluigi.edu.list.SinglyLinkedList.getViewPosition(int)}}\pysiglinewithargsret{public {\hyperref[\detokenize{com/linuxluigi/edu/data/ViewPosition:com.linuxluigi.edu.data.ViewPosition}]{\sphinxcrossref{ViewPosition}}} \sphinxbfcode{getViewPosition}}{int\sphinxstyleemphasis{ index}}{}
Gibt das ViewPosition objekt des Knoten mit dem Index index zurück
\begin{quote}\begin{description}
\item[{Parameter}] \leavevmode\begin{itemize}
\item {} 
\sphinxstyleliteralstrong{index} -- Index des Knotens

\end{itemize}

\item[{Rückgabe}] \leavevmode
ViewPosition objekt des Knoten mit dem Index

\end{description}\end{quote}

\end{fulllineitems}



\subparagraph{getWith}
\label{\detokenize{com/linuxluigi/edu/list/SinglyLinkedList:getwith}}\index{getWith() (Java method)}

\begin{fulllineitems}
\phantomsection\label{\detokenize{com/linuxluigi/edu/list/SinglyLinkedList:com.linuxluigi.edu.list.SinglyLinkedList.getWith()}}\pysiglinewithargsret{public int \sphinxbfcode{getWith}}{}{}
Gibt die breite des Baumes in Pixel zurück
\begin{quote}\begin{description}
\item[{Rückgabe}] \leavevmode
breite des Baumes

\end{description}\end{quote}

\end{fulllineitems}



\subparagraph{isEmpty}
\label{\detokenize{com/linuxluigi/edu/list/SinglyLinkedList:isempty}}\index{isEmpty() (Java method)}

\begin{fulllineitems}
\phantomsection\label{\detokenize{com/linuxluigi/edu/list/SinglyLinkedList:com.linuxluigi.edu.list.SinglyLinkedList.isEmpty()}}\pysiglinewithargsret{public boolean \sphinxbfcode{isEmpty}}{}{}
Gibt zurück ob die Liste leer ist
\begin{quote}\begin{description}
\item[{Rückgabe}] \leavevmode
True == Liste ohne Knoten False == in der Liste sind Knoten enthalten

\end{description}\end{quote}

\end{fulllineitems}



\subparagraph{remove}
\label{\detokenize{com/linuxluigi/edu/list/SinglyLinkedList:remove}}\index{remove(int) (Java method)}

\begin{fulllineitems}
\phantomsection\label{\detokenize{com/linuxluigi/edu/list/SinglyLinkedList:com.linuxluigi.edu.list.SinglyLinkedList.remove(int)}}\pysiglinewithargsret{public void \sphinxbfcode{remove}}{int\sphinxstyleemphasis{ index}}{}
Löscht ein Knoten mit dem Index index
\begin{quote}\begin{description}
\item[{Parameter}] \leavevmode\begin{itemize}
\item {} 
\sphinxstyleliteralstrong{index} -- Index des zu löschenden Knotens

\end{itemize}

\end{description}\end{quote}

\end{fulllineitems}



\subparagraph{set}
\label{\detokenize{com/linuxluigi/edu/list/SinglyLinkedList:set}}\index{set(int, T) (Java method)}

\begin{fulllineitems}
\phantomsection\label{\detokenize{com/linuxluigi/edu/list/SinglyLinkedList:com.linuxluigi.edu.list.SinglyLinkedList.set(int, T)}}\pysiglinewithargsret{public void \sphinxbfcode{set}}{int\sphinxstyleemphasis{ index}, T\sphinxstyleemphasis{ data}}{}
Sichert ein Objekt in den Knoten mit den Index index
\begin{quote}\begin{description}
\item[{Parameter}] \leavevmode\begin{itemize}
\item {} 
\sphinxstyleliteralstrong{index} -- Index des Knotens

\item {} 
\sphinxstyleliteralstrong{data} -- zu sicherendes Objekt

\end{itemize}

\end{description}\end{quote}

\end{fulllineitems}



\subparagraph{setBinaryTreeFromList}
\label{\detokenize{com/linuxluigi/edu/list/SinglyLinkedList:setbinarytreefromlist}}\index{setBinaryTreeFromList(String{[}{]}{[}{]}) (Java method)}

\begin{fulllineitems}
\phantomsection\label{\detokenize{com/linuxluigi/edu/list/SinglyLinkedList:com.linuxluigi.edu.list.SinglyLinkedList.setBinaryTreeFromList(String____)}}\pysiglinewithargsret{public void \sphinxbfcode{setBinaryTreeFromList}}{\href{http://docs.oracle.com/javase/8/docs/api/java/lang/String.html}{String}{[}{]}{[}{]}\sphinxstyleemphasis{ binaryTreeArray}}{}
Konvertiert ein String{[}{]}{[}{]} in ein Binärbaum
\begin{quote}\begin{description}
\item[{Parameter}] \leavevmode\begin{itemize}
\item {} 
\sphinxstyleliteralstrong{binaryTreeArray} -- zu konvertierendendes String{[}{]}{[}{]}

\end{itemize}

\end{description}\end{quote}

\end{fulllineitems}



\subparagraph{sort}
\label{\detokenize{com/linuxluigi/edu/list/SinglyLinkedList:sort}}\index{sort(OrderBy) (Java method)}

\begin{fulllineitems}
\phantomsection\label{\detokenize{com/linuxluigi/edu/list/SinglyLinkedList:com.linuxluigi.edu.list.SinglyLinkedList.sort(OrderBy)}}\pysiglinewithargsret{public void \sphinxbfcode{sort}}{{\hyperref[\detokenize{com/linuxluigi/edu/list/OrderBy:com.linuxluigi.edu.list.OrderBy}]{\sphinxcrossref{OrderBy}}}\sphinxstyleemphasis{ orderBy}}{}
Sortiert den Baum nach ASC order DECS
\begin{quote}\begin{description}
\item[{Parameter}] \leavevmode\begin{itemize}
\item {} 
\sphinxstyleliteralstrong{orderBy} -- OrderBy.ASC == Sortieren nach ASC OrderBy.DESC == Sortieren nach DESC

\end{itemize}

\end{description}\end{quote}

\end{fulllineitems}



\subsection{com.linuxluigi.edu.view}
\label{\detokenize{com/linuxluigi/edu/view/package-index::doc}}\label{\detokenize{com/linuxluigi/edu/view/package-index:package-com.linuxluigi.edu.view}}\label{\detokenize{com/linuxluigi/edu/view/package-index:com-linuxluigi-edu-view}}\index{com.linuxluigi.edu.view (package)}

\subsubsection{DialogWindow}
\label{\detokenize{com/linuxluigi/edu/view/DialogWindow::doc}}\label{\detokenize{com/linuxluigi/edu/view/DialogWindow:dialogwindow}}\index{DialogWindow (Java class)}

\begin{fulllineitems}
\phantomsection\label{\detokenize{com/linuxluigi/edu/view/DialogWindow:com.linuxluigi.edu.view.DialogWindow}}\pysigline{public class \sphinxbfcode{DialogWindow} extends JFrame}
Dialog Fenster welches erscheint nachdem ein Knoten gedrückt worden ist. Welches folgenede Optionen liefert.
\begin{itemize}
\item {} 
Knoten hinzufügen

\item {} 
Knoten ändern

\item {} 
Knoten löschen

\end{itemize}

\end{fulllineitems}



\paragraph{Constructors}
\label{\detokenize{com/linuxluigi/edu/view/DialogWindow:constructors}}

\subparagraph{DialogWindow}
\label{\detokenize{com/linuxluigi/edu/view/DialogWindow:id1}}\index{DialogWindow(int, String) (Java constructor)}

\begin{fulllineitems}
\phantomsection\label{\detokenize{com/linuxluigi/edu/view/DialogWindow:com.linuxluigi.edu.view.DialogWindow.DialogWindow(int, String)}}\pysiglinewithargsret{public \sphinxbfcode{DialogWindow}}{int\sphinxstyleemphasis{ nodeId}, \href{http://docs.oracle.com/javase/8/docs/api/java/lang/String.html}{String}\sphinxstyleemphasis{ nodeContent}}{}
\end{fulllineitems}



\paragraph{Methods}
\label{\detokenize{com/linuxluigi/edu/view/DialogWindow:methods}}

\subparagraph{addAddListener}
\label{\detokenize{com/linuxluigi/edu/view/DialogWindow:addaddlistener}}\index{addAddListener(ActionListener) (Java method)}

\begin{fulllineitems}
\phantomsection\label{\detokenize{com/linuxluigi/edu/view/DialogWindow:com.linuxluigi.edu.view.DialogWindow.addAddListener(ActionListener)}}\pysiglinewithargsret{public void \sphinxbfcode{addAddListener}}{\href{http://docs.oracle.com/javase/8/docs/api/java/awt/event/ActionListener.html}{ActionListener}\sphinxstyleemphasis{ listenerForAddButton}}{}
Button Knoten hinzufügen
\begin{quote}\begin{description}
\item[{Parameter}] \leavevmode\begin{itemize}
\item {} 
\sphinxstyleliteralstrong{listenerForAddButton} -- ActionListener

\end{itemize}

\end{description}\end{quote}

\end{fulllineitems}



\subparagraph{addRemoveListener}
\label{\detokenize{com/linuxluigi/edu/view/DialogWindow:addremovelistener}}\index{addRemoveListener(ActionListener) (Java method)}

\begin{fulllineitems}
\phantomsection\label{\detokenize{com/linuxluigi/edu/view/DialogWindow:com.linuxluigi.edu.view.DialogWindow.addRemoveListener(ActionListener)}}\pysiglinewithargsret{public void \sphinxbfcode{addRemoveListener}}{\href{http://docs.oracle.com/javase/8/docs/api/java/awt/event/ActionListener.html}{ActionListener}\sphinxstyleemphasis{ listenerForRemoveButton}}{}
Button Knoten löschen
\begin{quote}\begin{description}
\item[{Parameter}] \leavevmode\begin{itemize}
\item {} 
\sphinxstyleliteralstrong{listenerForRemoveButton} -- ActionListener

\end{itemize}

\end{description}\end{quote}

\end{fulllineitems}



\subparagraph{addRenameListener}
\label{\detokenize{com/linuxluigi/edu/view/DialogWindow:addrenamelistener}}\index{addRenameListener(ActionListener) (Java method)}

\begin{fulllineitems}
\phantomsection\label{\detokenize{com/linuxluigi/edu/view/DialogWindow:com.linuxluigi.edu.view.DialogWindow.addRenameListener(ActionListener)}}\pysiglinewithargsret{public void \sphinxbfcode{addRenameListener}}{\href{http://docs.oracle.com/javase/8/docs/api/java/awt/event/ActionListener.html}{ActionListener}\sphinxstyleemphasis{ listenerForRenameButton}}{}
Button Knoten ändern
\begin{quote}\begin{description}
\item[{Parameter}] \leavevmode\begin{itemize}
\item {} 
\sphinxstyleliteralstrong{listenerForRenameButton} -- ActionListener

\end{itemize}

\end{description}\end{quote}

\end{fulllineitems}



\subparagraph{getNodeId}
\label{\detokenize{com/linuxluigi/edu/view/DialogWindow:getnodeid}}\index{getNodeId() (Java method)}

\begin{fulllineitems}
\phantomsection\label{\detokenize{com/linuxluigi/edu/view/DialogWindow:com.linuxluigi.edu.view.DialogWindow.getNodeId()}}\pysiglinewithargsret{public int \sphinxbfcode{getNodeId}}{}{}
Gibt die Knoten ID des DialogWindow zurück
\begin{quote}\begin{description}
\item[{Rückgabe}] \leavevmode
Knoten ID

\end{description}\end{quote}

\end{fulllineitems}



\subparagraph{getText}
\label{\detokenize{com/linuxluigi/edu/view/DialogWindow:gettext}}\index{getText() (Java method)}

\begin{fulllineitems}
\phantomsection\label{\detokenize{com/linuxluigi/edu/view/DialogWindow:com.linuxluigi.edu.view.DialogWindow.getText()}}\pysiglinewithargsret{public \href{http://docs.oracle.com/javase/8/docs/api/java/lang/String.html}{String} \sphinxbfcode{getText}}{}{}
Gibt das Textfeld des DialogWindows zurück
\begin{quote}\begin{description}
\item[{Rückgabe}] \leavevmode
Textfeld des Dialogfenster als String

\end{description}\end{quote}

\end{fulllineitems}



\subsubsection{NodePanel}
\label{\detokenize{com/linuxluigi/edu/view/NodePanel::doc}}\label{\detokenize{com/linuxluigi/edu/view/NodePanel:nodepanel}}\index{NodePanel (Java class)}

\begin{fulllineitems}
\phantomsection\label{\detokenize{com/linuxluigi/edu/view/NodePanel:com.linuxluigi.edu.view.NodePanel}}\pysigline{public class \sphinxbfcode{NodePanel} extends JPanel}
Ein JPanel welches die Binärbaumknoten als Button zeichnet und mit Strichen verbindet.

\end{fulllineitems}



\paragraph{Fields}
\label{\detokenize{com/linuxluigi/edu/view/NodePanel:fields}}

\subparagraph{jButtons}
\label{\detokenize{com/linuxluigi/edu/view/NodePanel:jbuttons}}\index{jButtons (Java field)}

\begin{fulllineitems}
\phantomsection\label{\detokenize{com/linuxluigi/edu/view/NodePanel:com.linuxluigi.edu.view.NodePanel.jButtons}}\pysigline{public JButton{[}{]} \sphinxbfcode{jButtons}}
\end{fulllineitems}



\subparagraph{jLabels}
\label{\detokenize{com/linuxluigi/edu/view/NodePanel:jlabels}}\index{jLabels (Java field)}

\begin{fulllineitems}
\phantomsection\label{\detokenize{com/linuxluigi/edu/view/NodePanel:com.linuxluigi.edu.view.NodePanel.jLabels}}\pysigline{public JLabel{[}{]} \sphinxbfcode{jLabels}}
\end{fulllineitems}



\paragraph{Constructors}
\label{\detokenize{com/linuxluigi/edu/view/NodePanel:constructors}}

\subparagraph{NodePanel}
\label{\detokenize{com/linuxluigi/edu/view/NodePanel:id1}}\index{NodePanel() (Java constructor)}

\begin{fulllineitems}
\phantomsection\label{\detokenize{com/linuxluigi/edu/view/NodePanel:com.linuxluigi.edu.view.NodePanel.NodePanel()}}\pysiglinewithargsret{public \sphinxbfcode{NodePanel}}{}{}
Konstruktor, setz das Layout zu null

\end{fulllineitems}



\paragraph{Methods}
\label{\detokenize{com/linuxluigi/edu/view/NodePanel:methods}}

\subparagraph{addNodeListener}
\label{\detokenize{com/linuxluigi/edu/view/NodePanel:addnodelistener}}\index{addNodeListener(ActionListener) (Java method)}

\begin{fulllineitems}
\phantomsection\label{\detokenize{com/linuxluigi/edu/view/NodePanel:com.linuxluigi.edu.view.NodePanel.addNodeListener(ActionListener)}}\pysiglinewithargsret{public void \sphinxbfcode{addNodeListener}}{\href{http://docs.oracle.com/javase/8/docs/api/java/awt/event/ActionListener.html}{ActionListener}\sphinxstyleemphasis{ listenerForNodeButton}}{}
Actionlistener für alle Knoten
\begin{quote}\begin{description}
\item[{Parameter}] \leavevmode\begin{itemize}
\item {} 
\sphinxstyleliteralstrong{listenerForNodeButton} -- 

\end{itemize}

\end{description}\end{quote}

\end{fulllineitems}



\subparagraph{getJPanel}
\label{\detokenize{com/linuxluigi/edu/view/NodePanel:getjpanel}}\index{getJPanel(Listlabel) (Java method)}

\begin{fulllineitems}
\phantomsection\label{\detokenize{com/linuxluigi/edu/view/NodePanel:com.linuxluigi.edu.view.NodePanel.getJPanel(Listlabel)}}\pysiglinewithargsret{public JPanel \sphinxbfcode{getJPanel}}{{\hyperref[\detokenize{com/linuxluigi/edu/list/Listlabel:com.linuxluigi.edu.list.Listlabel}]{\sphinxcrossref{Listlabel}}}\textless{}{\hyperref[\detokenize{com/linuxluigi/edu/data/NodeData:com.linuxluigi.edu.data.NodeData}]{\sphinxcrossref{NodeData}}}\textgreater{}\sphinxstyleemphasis{ nodeList}}{}
Aktuallesiert das JPanel mithilfe des neuen Binärbaumes
\begin{quote}\begin{description}
\item[{Parameter}] \leavevmode\begin{itemize}
\item {} 
\sphinxstyleliteralstrong{nodeList} -- neuer Binärbaum

\end{itemize}

\item[{Rückgabe}] \leavevmode
Gibt das aktuallesierte JPanel zurück

\end{description}\end{quote}

\end{fulllineitems}



\subparagraph{paintComponent}
\label{\detokenize{com/linuxluigi/edu/view/NodePanel:paintcomponent}}\index{paintComponent(Graphics) (Java method)}

\begin{fulllineitems}
\phantomsection\label{\detokenize{com/linuxluigi/edu/view/NodePanel:com.linuxluigi.edu.view.NodePanel.paintComponent(Graphics)}}\pysiglinewithargsret{protected void \sphinxbfcode{paintComponent}}{Graphics\sphinxstyleemphasis{ g}}{}
Zeichnet alle Linien
\begin{quote}\begin{description}
\item[{Parameter}] \leavevmode\begin{itemize}
\item {} 
\sphinxstyleliteralstrong{g} -- 

\end{itemize}

\end{description}\end{quote}

\end{fulllineitems}



\subsubsection{View}
\label{\detokenize{com/linuxluigi/edu/view/View:view}}\label{\detokenize{com/linuxluigi/edu/view/View::doc}}\index{View (Java class)}

\begin{fulllineitems}
\phantomsection\label{\detokenize{com/linuxluigi/edu/view/View:com.linuxluigi.edu.view.View}}\pysigline{public class \sphinxbfcode{View} extends JFrame}
Main View, innerhalb dieser View wird das Hauptmenü und Knoten Zeichnung dargestellt.

\end{fulllineitems}



\paragraph{Fields}
\label{\detokenize{com/linuxluigi/edu/view/View:fields}}

\subparagraph{jScrollPane}
\label{\detokenize{com/linuxluigi/edu/view/View:jscrollpane}}\index{jScrollPane (Java field)}

\begin{fulllineitems}
\phantomsection\label{\detokenize{com/linuxluigi/edu/view/View:com.linuxluigi.edu.view.View.jScrollPane}}\pysigline{ JScrollPane \sphinxbfcode{jScrollPane}}
\end{fulllineitems}



\paragraph{Constructors}
\label{\detokenize{com/linuxluigi/edu/view/View:constructors}}

\subparagraph{View}
\label{\detokenize{com/linuxluigi/edu/view/View:id1}}\index{View(int, int) (Java constructor)}

\begin{fulllineitems}
\phantomsection\label{\detokenize{com/linuxluigi/edu/view/View:com.linuxluigi.edu.view.View.View(int, int)}}\pysiglinewithargsret{public \sphinxbfcode{View}}{int\sphinxstyleemphasis{ with}, int\sphinxstyleemphasis{ height}}{}
Konstruktor der View
\begin{itemize}
\item {} 
Setz den Titel der View

\item {} 
Erstellt die Menü Bar

\item {} 
Schaltet sich selbst sichtbar

\end{itemize}
\begin{quote}\begin{description}
\item[{Parameter}] \leavevmode\begin{itemize}
\item {} 
\sphinxstyleliteralstrong{with} -- Breite des View Fensters in Pixel

\item {} 
\sphinxstyleliteralstrong{height} -- Höhe des View Fensters in Pixel

\end{itemize}

\end{description}\end{quote}

\end{fulllineitems}



\paragraph{Methods}
\label{\detokenize{com/linuxluigi/edu/view/View:methods}}

\subparagraph{addMenuExitListener}
\label{\detokenize{com/linuxluigi/edu/view/View:addmenuexitlistener}}\index{addMenuExitListener(ActionListener) (Java method)}

\begin{fulllineitems}
\phantomsection\label{\detokenize{com/linuxluigi/edu/view/View:com.linuxluigi.edu.view.View.addMenuExitListener(ActionListener)}}\pysiglinewithargsret{public void \sphinxbfcode{addMenuExitListener}}{\href{http://docs.oracle.com/javase/8/docs/api/java/awt/event/ActionListener.html}{ActionListener}\sphinxstyleemphasis{ listenerForMenuExit}}{}
Erstellt den Actionlistener für: Exit
\begin{quote}\begin{description}
\item[{Parameter}] \leavevmode\begin{itemize}
\item {} 
\sphinxstyleliteralstrong{listenerForMenuExit} -- ActionListener

\end{itemize}

\end{description}\end{quote}

\end{fulllineitems}



\subparagraph{addMenuLoadListener}
\label{\detokenize{com/linuxluigi/edu/view/View:addmenuloadlistener}}\index{addMenuLoadListener(ActionListener) (Java method)}

\begin{fulllineitems}
\phantomsection\label{\detokenize{com/linuxluigi/edu/view/View:com.linuxluigi.edu.view.View.addMenuLoadListener(ActionListener)}}\pysiglinewithargsret{public void \sphinxbfcode{addMenuLoadListener}}{\href{http://docs.oracle.com/javase/8/docs/api/java/awt/event/ActionListener.html}{ActionListener}\sphinxstyleemphasis{ listenerForMenuLoad}}{}
Erstellt den Actionlistener für: Menu - Binärbaum von Json laden
\begin{quote}\begin{description}
\item[{Parameter}] \leavevmode\begin{itemize}
\item {} 
\sphinxstyleliteralstrong{listenerForMenuLoad} -- ActionListener

\end{itemize}

\end{description}\end{quote}

\end{fulllineitems}



\subparagraph{addMenuNewListener}
\label{\detokenize{com/linuxluigi/edu/view/View:addmenunewlistener}}\index{addMenuNewListener(ActionListener) (Java method)}

\begin{fulllineitems}
\phantomsection\label{\detokenize{com/linuxluigi/edu/view/View:com.linuxluigi.edu.view.View.addMenuNewListener(ActionListener)}}\pysiglinewithargsret{public void \sphinxbfcode{addMenuNewListener}}{\href{http://docs.oracle.com/javase/8/docs/api/java/awt/event/ActionListener.html}{ActionListener}\sphinxstyleemphasis{ listenerForMenuNew}}{}
Erstellt den Actionlistener für: Menu - Neuen Binärbaum anlegen
\begin{quote}\begin{description}
\item[{Parameter}] \leavevmode\begin{itemize}
\item {} 
\sphinxstyleliteralstrong{listenerForMenuNew} -- ActionListener

\end{itemize}

\end{description}\end{quote}

\end{fulllineitems}



\subparagraph{addMenuSaveListener}
\label{\detokenize{com/linuxluigi/edu/view/View:addmenusavelistener}}\index{addMenuSaveListener(ActionListener) (Java method)}

\begin{fulllineitems}
\phantomsection\label{\detokenize{com/linuxluigi/edu/view/View:com.linuxluigi.edu.view.View.addMenuSaveListener(ActionListener)}}\pysiglinewithargsret{public void \sphinxbfcode{addMenuSaveListener}}{\href{http://docs.oracle.com/javase/8/docs/api/java/awt/event/ActionListener.html}{ActionListener}\sphinxstyleemphasis{ listenerForMenuSave}}{}
Erstellt den Actionlistener für: Menu - Binärbaum in Json speichern
\begin{quote}\begin{description}
\item[{Parameter}] \leavevmode\begin{itemize}
\item {} 
\sphinxstyleliteralstrong{listenerForMenuSave} -- ActionListener

\end{itemize}

\end{description}\end{quote}

\end{fulllineitems}



\subparagraph{addNodeListener}
\label{\detokenize{com/linuxluigi/edu/view/View:addnodelistener}}\index{addNodeListener(ActionListener) (Java method)}

\begin{fulllineitems}
\phantomsection\label{\detokenize{com/linuxluigi/edu/view/View:com.linuxluigi.edu.view.View.addNodeListener(ActionListener)}}\pysiglinewithargsret{public void \sphinxbfcode{addNodeListener}}{\href{http://docs.oracle.com/javase/8/docs/api/java/awt/event/ActionListener.html}{ActionListener}\sphinxstyleemphasis{ listenerForNodeButton}}{}
Erstellt den Actionlistener für: Alle Knoten im Binärbaum.
\begin{quote}\begin{description}
\item[{Parameter}] \leavevmode\begin{itemize}
\item {} 
\sphinxstyleliteralstrong{listenerForNodeButton} -- ActionListener

\end{itemize}

\end{description}\end{quote}

\end{fulllineitems}



\subparagraph{addSortAcsListener}
\label{\detokenize{com/linuxluigi/edu/view/View:addsortacslistener}}\index{addSortAcsListener(ActionListener) (Java method)}

\begin{fulllineitems}
\phantomsection\label{\detokenize{com/linuxluigi/edu/view/View:com.linuxluigi.edu.view.View.addSortAcsListener(ActionListener)}}\pysiglinewithargsret{public void \sphinxbfcode{addSortAcsListener}}{\href{http://docs.oracle.com/javase/8/docs/api/java/awt/event/ActionListener.html}{ActionListener}\sphinxstyleemphasis{ listenerForSortAcs}}{}
Erstellt den Actionlistener für: Menu - Binärbaum nach ACS sortieren
\begin{quote}\begin{description}
\item[{Parameter}] \leavevmode\begin{itemize}
\item {} 
\sphinxstyleliteralstrong{listenerForSortAcs} -- ActionListener

\end{itemize}

\end{description}\end{quote}

\end{fulllineitems}



\subparagraph{addSortDecsListener}
\label{\detokenize{com/linuxluigi/edu/view/View:addsortdecslistener}}\index{addSortDecsListener(ActionListener) (Java method)}

\begin{fulllineitems}
\phantomsection\label{\detokenize{com/linuxluigi/edu/view/View:com.linuxluigi.edu.view.View.addSortDecsListener(ActionListener)}}\pysiglinewithargsret{public void \sphinxbfcode{addSortDecsListener}}{\href{http://docs.oracle.com/javase/8/docs/api/java/awt/event/ActionListener.html}{ActionListener}\sphinxstyleemphasis{ listenerForSortDecs}}{}
Erstellt den Actionlistener für: Menu - Binärbaum nach DECS sortieren
\begin{quote}\begin{description}
\item[{Parameter}] \leavevmode\begin{itemize}
\item {} 
\sphinxstyleliteralstrong{listenerForSortDecs} -- ActionListener

\end{itemize}

\end{description}\end{quote}

\end{fulllineitems}



\subparagraph{setBinaryTree}
\label{\detokenize{com/linuxluigi/edu/view/View:setbinarytree}}\index{setBinaryTree(Listlabel) (Java method)}

\begin{fulllineitems}
\phantomsection\label{\detokenize{com/linuxluigi/edu/view/View:com.linuxluigi.edu.view.View.setBinaryTree(Listlabel)}}\pysiglinewithargsret{public void \sphinxbfcode{setBinaryTree}}{{\hyperref[\detokenize{com/linuxluigi/edu/list/Listlabel:com.linuxluigi.edu.list.Listlabel}]{\sphinxcrossref{Listlabel}}}\textless{}{\hyperref[\detokenize{com/linuxluigi/edu/data/NodeData:com.linuxluigi.edu.data.NodeData}]{\sphinxcrossref{NodeData}}}\textgreater{}\sphinxstyleemphasis{ nodeList}}{}
Den Binärbaum updaten und anschließend wird diese View neu gezeichnet.
\begin{quote}\begin{description}
\item[{Parameter}] \leavevmode\begin{itemize}
\item {} 
\sphinxstyleliteralstrong{nodeList} -- Binärbaum im Listenformat

\end{itemize}

\end{description}\end{quote}

\end{fulllineitems}



\chapter{Sonstiges}
\label{\detokenize{index:sonstiges}}

\section{Lizenz}
\label{\detokenize{license::doc}}\label{\detokenize{license:lizenz}}
\index{Lizenz}
MIT License

Copyright (c) 2017 Steffen Exler

Hiermit wird unentgeltlich jeder Person, die eine Kopie der Software und der zugehörigen Dokumentationen (die ``Software'') erhält, die Erlaubnis erteilt, sie uneingeschränkt zu nutzen, inklusive und ohne Ausnahme mit dem Recht, sie zu verwenden, zu kopieren, zu verändern, zusammenzufügen, zu veröffentlichen, zu verbreiten, zu unterlizenzieren und/oder zu verkaufen, und Personen, denen diese Software überlassen wird, diese Rechte zu verschaffen, unter den folgenden Bedingungen:

Der obige Urheberrechtsvermerk und dieser Erlaubnisvermerk sind in allen Kopien oder Teilkopien der Software beizulegen.

DIE SOFTWARE WIRD OHNE JEDE AUSDRÜCKLICHE ODER IMPLIZIERTE GARANTIE BEREITGESTELLT, EINSCHLIEßLICH DER GARANTIE ZUR BENUTZUNG FÜR DEN VORGESEHENEN ODER EINEM BESTIMMTEN ZWECK SOWIE JEGLICHER RECHTSVERLETZUNG, JEDOCH NICHT DARAUF BESCHRÄNKT. IN KEINEM FALL SIND DIE AUTOREN ODER COPYRIGHTINHABER FÜR JEGLICHEN SCHADEN ODER SONSTIGE ANSPRÜCHE HAFTBAR ZU MACHEN, OB INFOLGE DER ERFÜLLUNG EINES VERTRAGES, EINES DELIKTES ODER ANDERS IM ZUSAMMENHANG MIT DER SOFTWARE ODER SONSTIGER VERWENDUNG DER SOFTWARE ENTSTANDEN.


\section{Kontakt}
\label{\detokenize{license:kontakt}}
\index{Kontakt}
Fragen? Kontaktieren sie \href{mailto:Steffen.Exler@gmail.com}{Steffen.Exler@gmail.com}


\section{Hilfe}
\label{\detokenize{help::doc}}\label{\detokenize{help:hilfe}}
\index{Support}
Wenn Sie hilfe brauchen email \href{mailto:Steffen.Exler@gmail.com}{Steffen.Exler@gmail.com}


\chapter{Indices and tables}
\label{\detokenize{index:indices-and-tables}}\begin{itemize}
\item {} 
\DUrole{xref,std,std-ref}{genindex}

\item {} 
\DUrole{xref,std,std-ref}{modindex}

\item {} 
\DUrole{xref,std,std-ref}{search}

\end{itemize}



\renewcommand{\indexname}{Stichwortverzeichnis}
\printindex
\end{document}